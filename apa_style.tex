% Options for packages loaded elsewhere
\PassOptionsToPackage{unicode}{hyperref}
\PassOptionsToPackage{hyphens}{url}
%
\documentclass[
  english,
  man]{apa6}
\usepackage{lmodern}
\usepackage{amssymb,amsmath}
\usepackage{ifxetex,ifluatex}
\ifnum 0\ifxetex 1\fi\ifluatex 1\fi=0 % if pdftex
  \usepackage[T1]{fontenc}
  \usepackage[utf8]{inputenc}
  \usepackage{textcomp} % provide euro and other symbols
\else % if luatex or xetex
  \usepackage{unicode-math}
  \defaultfontfeatures{Scale=MatchLowercase}
  \defaultfontfeatures[\rmfamily]{Ligatures=TeX,Scale=1}
\fi
% Use upquote if available, for straight quotes in verbatim environments
\IfFileExists{upquote.sty}{\usepackage{upquote}}{}
\IfFileExists{microtype.sty}{% use microtype if available
  \usepackage[]{microtype}
  \UseMicrotypeSet[protrusion]{basicmath} % disable protrusion for tt fonts
}{}
\makeatletter
\@ifundefined{KOMAClassName}{% if non-KOMA class
  \IfFileExists{parskip.sty}{%
    \usepackage{parskip}
  }{% else
    \setlength{\parindent}{0pt}
    \setlength{\parskip}{6pt plus 2pt minus 1pt}}
}{% if KOMA class
  \KOMAoptions{parskip=half}}
\makeatother
\usepackage{xcolor}
\IfFileExists{xurl.sty}{\usepackage{xurl}}{} % add URL line breaks if available
\IfFileExists{bookmark.sty}{\usepackage{bookmark}}{\usepackage{hyperref}}
\hypersetup{
  pdftitle={An interdisciplinary examination of stress and injury occurrence in athletes},
  pdflang={en-EN},
  pdfkeywords={Sports Injury, Stress, Interdisciplinary, Bayesian Network},
  hidelinks,
  pdfcreator={LaTeX via pandoc}}
\urlstyle{same} % disable monospaced font for URLs
\usepackage{graphicx}
\makeatletter
\def\maxwidth{\ifdim\Gin@nat@width>\linewidth\linewidth\else\Gin@nat@width\fi}
\def\maxheight{\ifdim\Gin@nat@height>\textheight\textheight\else\Gin@nat@height\fi}
\makeatother
% Scale images if necessary, so that they will not overflow the page
% margins by default, and it is still possible to overwrite the defaults
% using explicit options in \includegraphics[width, height, ...]{}
\setkeys{Gin}{width=\maxwidth,height=\maxheight,keepaspectratio}
% Set default figure placement to htbp
\makeatletter
\def\fps@figure{htbp}
\makeatother
\setlength{\emergencystretch}{3em} % prevent overfull lines
\providecommand{\tightlist}{%
  \setlength{\itemsep}{0pt}\setlength{\parskip}{0pt}}
\setcounter{secnumdepth}{-\maxdimen} % remove section numbering
% Make \paragraph and \subparagraph free-standing
\ifx\paragraph\undefined\else
  \let\oldparagraph\paragraph
  \renewcommand{\paragraph}[1]{\oldparagraph{#1}\mbox{}}
\fi
\ifx\subparagraph\undefined\else
  \let\oldsubparagraph\subparagraph
  \renewcommand{\subparagraph}[1]{\oldsubparagraph{#1}\mbox{}}
\fi
% Manuscript styling
\usepackage{upgreek}
\captionsetup{font=singlespacing,justification=justified}

% Table formatting
\usepackage{longtable}
\usepackage{lscape}
% \usepackage[counterclockwise]{rotating}   % Landscape page setup for large tables
\usepackage{multirow}		% Table styling
\usepackage{tabularx}		% Control Column width
\usepackage[flushleft]{threeparttable}	% Allows for three part tables with a specified notes section
\usepackage{threeparttablex}            % Lets threeparttable work with longtable

% Create new environments so endfloat can handle them
% \newenvironment{ltable}
%   {\begin{landscape}\begin{center}\begin{threeparttable}}
%   {\end{threeparttable}\end{center}\end{landscape}}
\newenvironment{lltable}{\begin{landscape}\begin{center}\begin{ThreePartTable}}{\end{ThreePartTable}\end{center}\end{landscape}}

% Enables adjusting longtable caption width to table width
% Solution found at http://golatex.de/longtable-mit-caption-so-breit-wie-die-tabelle-t15767.html
\makeatletter
\newcommand\LastLTentrywidth{1em}
\newlength\longtablewidth
\setlength{\longtablewidth}{1in}
\newcommand{\getlongtablewidth}{\begingroup \ifcsname LT@\roman{LT@tables}\endcsname \global\longtablewidth=0pt \renewcommand{\LT@entry}[2]{\global\advance\longtablewidth by ##2\relax\gdef\LastLTentrywidth{##2}}\@nameuse{LT@\roman{LT@tables}} \fi \endgroup}

% \setlength{\parindent}{0.5in}
% \setlength{\parskip}{0pt plus 0pt minus 0pt}

% \usepackage{etoolbox}
\makeatletter
\patchcmd{\HyOrg@maketitle}
  {\section{\normalfont\normalsize\abstractname}}
  {\section*{\normalfont\normalsize\abstractname}}
  {}{\typeout{Failed to patch abstract.}}
\makeatother
\shorttitle{Stress and sports injury}
\author{Harry Fisher\textsuperscript{1}, Dr Marianne Gittoes\textsuperscript{1}, Professor Lynne Evans\textsuperscript{1}, Miss Leah Bitchell\textsuperscript{1}, Dr Richard Mullen\textsuperscript{2}, \& Dr Marco Scutari\textsuperscript{3}}
\affiliation{
\vspace{0.5cm}
\textsuperscript{1} Cardiff Metropolitan University, Cardiff, United Kingdom\\\textsuperscript{2} Brunel University, London, United Kingdom\\\textsuperscript{3} Istituto Dalle Molle di Studi sull'Intelligenza Artificiale (IDSIA), Manno, Switzerland}
\authornote{

Correspondence concerning this article should be addressed to Harry Fisher, Cyncoed Road, Cardiff, CF23 6XD. E-mail: harryfisher21@gmail.com}
\keywords{Sports Injury, Stress, Interdisciplinary, Bayesian Network}
\DeclareDelayedFloatFlavor{ThreePartTable}{table}
\DeclareDelayedFloatFlavor{lltable}{table}
\DeclareDelayedFloatFlavor*{longtable}{table}
\makeatletter
\renewcommand{\efloat@iwrite}[1]{\immediate\expandafter\protected@write\csname efloat@post#1\endcsname{}}
\makeatother
\usepackage{lineno}

\linenumbers
\usepackage{csquotes}
\ifxetex
  % Load polyglossia as late as possible: uses bidi with RTL langages (e.g. Hebrew, Arabic)
  \usepackage{polyglossia}
  \setmainlanguage[]{english}
\else
  \usepackage[shorthands=off,main=english]{babel}
\fi
\newlength{\cslhangindent}
\setlength{\cslhangindent}{1.5em}
\newenvironment{cslreferences}%
  {\setlength{\parindent}{0pt}%
  \everypar{\setlength{\hangindent}{\cslhangindent}}\ignorespaces}%
  {\par}

\title{An interdisciplinary examination of stress and injury occurrence in athletes}

\date{}

\abstract{
This paper adopts a novel, interdisciplinary approach to explore the relationship between psychosocial factors, physiological stress-related markers and occurrence of injury in athletes using a repeated measures design across a 2-year data collection period. At three data collection time-points, athletes completed measures of major life events, the reinforcement sensitivity theory personality questionnaire, muscle stiffness, heart rate variability and postural stability, and reported any injuries they had sustained since the last data collection. Two Bayesian networks were used to examine the relationships between variables and model the changes between data collection points in the study. Findings revealed muscle stiffness to have the strongest relationship with injury occurrence, with high levels of stiffness increasing the probability of sustaining an injury. Negative life events did not increase the probability of injury occurrence at any single time-point; however, when examining changes between time points, increases in negative life events did increase the probability of injury. In addition, the combination of increases in negative life events and muscle stiffness resulted in the greatest probability of sustaining an injury. Findings demonstrated the importance of both an interdisciplinary approach and a repeated measures design to furthering our understanding of the relationship between stress-related markers and injury occurrence.
}

\begin{document}
\maketitle

\section{Introduction}

Over the last four decades sport-related injuries have received increased research attention (Ivarsson et al., 2017) in response to the high incidence (Rosa et al., 2014; Sheu et al., 2016) and associated undesirable physical and psychological effects (Leddy et al., 1994; Brewer, 2012). Multiple psychological (Slimani et al., 2018), anatomical (Murphy et al., 2003), biomechanical (Neely, 1998; Hughes, 2014) and environmental (Meeuwisse et al., 2007) factors have been associated with sports injury occurrence and several models of injury causation have been proposed that highlight the multifactorial nature of injury occurrence (Kumar, 2001; Meeuwisse et al., 2007; Wiese-Bjornstal, 2009), of which one of the most widely cited was developed by Williams and Anderson (Fig \ref{fig:fig1}; Andersen and Williams, 1988; Williams and Andersen, 1998).

Williams and Andersen's (Williams and Andersen, 1998) stress-injury model proposed that when faced with a potentially stressful athletic situation, an athlete's personality traits (e.g., hardiness, locus of control and competitive trait anxiety), history of stressors (e.g., major life events and previous injuries) and coping resources (e.g., general coping behaviours) contribute to the injury response, either interactively or in isolation. The stress response is central to the model and reflects the bi-directional relationship between athletes' appraisal of, and response to, a stressful athletic situation. The model predicts that athletes who have a history of stressors, personality traits that intensify the stress response and few coping resources, will exhibit greater attentional (e.g., peripheral narrowing) and/or physiological (e.g., increased muscle tension) responses that put these individuals at greater injury risk.

Within Williams and Andersen's (Williams and Andersen, 1998) model, major life events, which is a component of an athlete's history of stressors, most consistently predicts injury occurrence (Williams and Andersen, 2007); specifically, major life events with a negative, as opposed to positive, valence (Passer and Seese, 1983; Maddison and Prapavessis, 2005). Personality traits and coping resources have also been found to predict injury with athletes more likely to sustain an injury if they have poor social support and psychological coping skills, and high trait anxiety and elevated competitive state anxiety; compared to those with the opposing profile (Smith et al., 1990; Lavallée and Flint, 1996; Ivarsson and Johnson, 2010).However, the amount of variance explained by these psychosocial factors has been modest and typically between 5 - 30\% (Galambos et al., 2005; Ivarsson and Johnson, 2010), which indicates a likely interaction with other factors.

While the psychosocial factors in Williams and Andersen's (Williams and Andersen, 1998) model have received the most research attention, less insight into the mechanisms through which these factors are proposed to exert their effect exists. To elaborate, the model suggests that injuries are likely to occur through either increased physiological arousal resulting in increased muscle tension and reduced flexibility or attentional deficits caused by increased distractibility and peripheral narrowing. However, to date, the research has largely focused on attentional deficits (Andersen and Williams, 1999; Rogers and Landers, 2005; Wilkerson, 2012; Swanik et al., 2007). For example, Andersen and Williams (Andersen and Williams, 1999) found athletes with high life event stress coupled with low social support had greater peripheral narrowing under stressful conditions compared to athletes with the opposing profile; these athletes went on to sustain an increased number of injuries during the following athletic season. Rodgers and Landers (Rogers and Landers, 2005) further supported Andersen and Williams's (Andersen and Williams, 1999) earlier findings by identifying that peripheral narrowing under stress mediated 8.1\% of the relationship between negative life events and injury.

Knowledge of the physiological factors (e.g., increased muscle tension and reduced motor control) contributing to the remaining variance between negative life events and athletic injury remains sparse (cf.~Williams and Andersen, 1998). One challenge faced by researchers addressing the sports injury problem through a psychological lens is the multifactorial nature of injury, and the possible interaction with physiological factors in the stress response (Meeuwisse et al., 2007; Wiese-Bjornstal, 2009). For example, a large body of research has suggested that training-related stress is also likely to contribute to the stress response and injury occurrence (Lee et al., 2017; Djaoui et al., 2017) and may account for the unexplained variance from the psychological predictors. Appaneal and Perna (Appaneal and Perna, 2014) proposed the biopsychosocial model of stress athletic injury and health (BMSAIH) to serve as an extension to Williams and Andersen's (Williams and Andersen, 1998) model. The BMSAIH enhances our understanding of the mediating pathways between the stress response and injury alongside other health outcomes and behavioral factors that impact sports participation (Appaneal and Perna, 2014). The central tenet of the BMSAIH is that psychosocial distress (e.g., negative life events) may act synergistically with training-related stress as a result of high-intensity and high-volume sports training, and ``widen the window of susceptibility'' (Appaneal and Perna, 2014, 74) to a range of undesirable health outcomes including illness and injury. Consequently, the BMSAIH provides an important framework that has enhanced insight into the multi-faceted nature of the injury process by building on Williams and Andersen's (Williams and Andersen, 1998) model whilst including other physiological markers of training-related stress.

Although research supporting the BMSAIH has mainly focused on the relationship between hormonal responses to training and injury occurrence (Perna and McDowell, 1995; Perna et al., 1997, 2003), other training-related stress markers including heart rate variability (Bellenger et al., 2016; Williams et al., 2017), muscle stiffness (Pruyn et al., 2015) and postural stability (Romero-Franco et al., 2014) have been reported to be associated with an increased risk of injury. However, the largely mono-disciplinary approach to examining the role of these markers has provided limited insight into the multifaceted interactions with psychological factors that may contribute to injury occurrence. Furthermore, a reliance on designs that capture a single point of measurement precludes the assessment of intra as well as inter-individual changes and the effect of the time interval between measurement and injury occurrence on subsequent injuries (Johnson et al., 2014). Such an approach fails to capture changes in both psychological factors and physiological markers that may occur preceding an injury. Importantly, a repeated measurement approach would enable an assessing of how variables and their interactions change with respect to time provide greater insight into the effect that repeated exposure to major life events and other stress-related factors has on injury occurrence.

Recently, Bittencourt et al.~(Bittencourt et al., 2016) advocated a move away from studying isolated risk factors and instead, adopt a complex systems approach in order to understand injury occurrence. Such an approach posits that injury may arise from a complex ``web of determinants'' (Bittencourt et al., 2016, 3), where different factors interact in unpredictable and unplanned ways, but result in a global outcome pattern of either adaptation or injury. Capturing the uncertainty and complexity of the relationships between different variables using an appropriate interdisciplinary analysis within the framework of a complex systems approach is challenging. Bayesian network (BN) modelling provides one solution by allowing the construction of graphical probabilistic models using the underlying structure that connects different variables (Scutari and Denis, 2014). The learned BN structure can be used for inference by obtaining the posterior probabilities of a particular variable for a given query (e.g., if the value of variable A is x and the value of variable B is y, what is the probability variable C of being value z?). Furthermore, unlike regression or structural equation models, BN's do not distinguish between dependent and independent variables when the underlying relationship in the network may not be known (Olmedilla et al., 2018). BN modelling subsequently provides a valuable but underused interdisciplinary approach to investigating the complex and unpredictable interactions of psychological and physiological factors implicated in the injury process.

Using the frameworks provided by Williams and Andersen's (Williams and Andersen, 1998) stress injury model and Appaneal's BMSAIH model, the aim of this interdisciplinary study was to develop new understanding of the multifaceted interactions of psychological and physiological stress-related factors with injury occurrence. A prospective, repeated measures design incorporating field-based physiological and psychological measures combined with a BN modelling analysis was used to address the study aim.

\section{Methods}

\subsection{Participants}

A total of 351 athletes (male: \emph{n} = 231, female: \emph{n} = 120) were initially recruited for the study from a British University and local sports clubs (Table \ref{tab:table1}). Participants had an average age of 22.0 \(\pm\) 7.0 years and represented a range of team (football, rugby, netball, cricket, lacrosse, basketball and field hockey) and individual sports (athletics, tennis, weightlifting, gymnastics, judo, swimming and golf.
Participants self-rated competitive level ranged from recreational to international standard.
They were engaged in training for their respective sports for at least five hours per week.
A total of 126 (49.03\%) participants had sustained an injury in the 12 months prior to the start of the study (male: \emph{n} = 114 {[}49\%{]}, female; \emph{n} = 48 {[}40\%{]}).
All participants were injury free (no modifications to their usual training routine due to a sport related medical problem for a minimum of four weeks) at the start of the study.
Ethical approval was obtained from the University ethics committee prior to the start of the study and all participants provided informed consent.

\begin{table}[H]

\caption{\label{tab:table1}Participant characteristics.}
\centering
\begin{tabular}[t]{c|c|c}
\hline
\textbf{ } & \textbf{Female (n = 120)} & \textbf{Male (n = 231)}\\
\hline
\multicolumn{3}{l}{\textbf{Demographics M (SD)}}\\
\hline
\hspace{1em}Age (yrs) & 26.0 (11.3) & 20.2 (1.8)\\
\hline
\hspace{1em}Height (cm) & 167.4 (7.6) & 177.8 (7.8)\\
\hline
\hspace{1em}Body mass (kg) & 67.0 (9.5) & 82.0 (14.6)\\
\hline
\hspace{1em}Hours per week training & 8.5 (4.5) & 11.2 (8.8)\\
\hline
\multicolumn{3}{l}{\textbf{Current competitive level n (\%)}}\\
\hline
\hspace{1em}Recreational & 3 (4) & 7 (4)\\
\hline
\hspace{1em}University & 45 (56) & 141 (80)\\
\hline
\hspace{1em}National/International & 33 (41) & 28 (16)\\
\hline
\end{tabular}
\end{table}

\hypertarget{measures}{%
\subsection{Measures}\label{measures}}

\hypertarget{injury}{%
\subsubsection{Injury}\label{injury}}

Participants self-reported any injuries they sustained at each data collection during the study period.
An injury was defined as any sports related medical problem causing the athlete to miss or modify their usual training routine for at least 24 hours (Fuller et al., 2006, 2007; Timpka et al., 2014).
Minor scrapes and bruises that may require certain modifications (e.g., strapping or taping) but did not limit continued participation were not considered injuries (cf.~Appaneal et al., 2009).
Injury status (did / did not sustain an injury) served as the main outcome measure.

\hypertarget{major-life-events}{%
\subsubsection{Major life events}\label{major-life-events}}

A modified version of the Life Events Survey for Collegiate Athletes (LESCA) was used to measure participants' history of life event stress (Petrie, 1992).
The LESCA is the most widely used measure of major life events for athletes in the sports injury literature.
Modifications were made to the LESCA to ensure the suitability of the items for the study cohort (S1 Table).
The LESCA comprises 69 items that reflect possible life events that participants may have experienced. Example items include, ``Major change in the frequency (increased or decreased) of social activities due to participation in sport'', ``Major change in the amount (more or less) of academic activity (homework, class time, etc)'' and ``Major change in level of athletic performance in actual competition (better or worse)''. Participants were asked to rate the perceived impact of each life event they had experienced within 12-months preceding the study onset on an 8-point Likert scale anchored at -4 (\(extremely\;negative\)) and +4 (\(extremely\;positive\)).
Negative and positive life event scores were calculated by summing the negative and positive scores, respectively. A score for total life events was also calculated by summing the absolute values for both negative and positive events.
Petrie (1992) (Petrie, 1992) reported test-retest reliabilities at 1-week and 8-weeks with values ranging from .76 to .84 (p \textless{} .001) and .48 to .72 (p \textless{} .001) respectively. Petrie also provided evidence of discriminant, convergent and predictive validity. For this study, composite reliability (Fornell and Larcker, 1981) was preferred to Cronbach's alpha as it does not assume parallelity (i.e., all factor loadings are constrained to be equal, and all error variances are constrained to be equal) and instead takes into consideration the varying factor loadings of the items in the questionnaire. The composite reliability for the LESCA in this study was .84.

\hypertarget{reinforcement-sensitivity-theory-personality-questionnaire}{%
\subsubsection{Reinforcement Sensitivity Theory Personality Questionnaire}\label{reinforcement-sensitivity-theory-personality-questionnaire}}

A revised version of the Reinforcement Sensitivity Theory Personality Questionnaire (RST-PQ) was used to measure motivation, emotion, personality and their relevance to psychopathology (Corr and Cooper, 2016).
The revised version of the RST-PQ comprises 51 statements that measure three major systems: Fight-Flight-Freeze System (FFFS; e.g., ``I am the sort of person who easily freezes-up when scared''), Behavioural Inhibition System (BIS; e.g., ``When trying to make a decision, I find myself constantly chewing it over'') and four Behavioural Approach System (BAS) factors; Reward Interest (e.g., ``I regularly try new activities just to see if I enjoy them''), Goal Drive Persistence (e.g., ``I am very persistent in achieving my goals''), Reward Reactivity (e.g., ``I get a special thrill when I am praised for something I've done well'') and Impulsivity (e.g., ``I find myself doing things on the spur of the moment'').
Participants rated each item on a scale from 1 (\(not\; at\; all\)) to 4 (\(highly\)) to reflect how well each statement described their personality in general.
The responses to items associated with each subscale (FFFS, BIS, RI, GDP, RR and I) were summed to give a total personality score that was subsequently used for further analysis.
The composite reliabilities for each subscale were; BIS = 0.92, FFFS = 0.77, GDP = 0.87, I = 0.71, RI = 0.77, RR = 0.81. Further details regarding the revised RST are in S1 Appendix.

\hypertarget{heart-rate-variability}{%
\subsubsection{Heart rate variability}\label{heart-rate-variability}}

A Polar V800 heart rate monitor (HRM) and Polar H7 Bluetooth chest strap (Polar OY, Finland) was used to collect inter-beat interval (IBI) data.
IBI recordings using the Polar V800 are highly comparable (ICC = 1.00) with ECG recordings (Giles et al., 2016), which are considered the gold standard for assessing HRV.
In addition, HRV indices calculated from IBI and ECG data have shown a strong correlation (r = .99) in athletes (Caminal et al., 2018) and under spontaneous breathing conditions (Plews et al., 2017).

\hypertarget{musculoskeletal-properties}{%
\subsubsection{Musculoskeletal properties}\label{musculoskeletal-properties}}

A handheld myometer (MyotonPRO, Myoton AS, Tallinn, Estonia) was used to measure passive muscle stiffness.
The MyotonPRO is a non-invasive, handheld device that applies a mechanical impulse of 0.40 N for 0.15 ms perpendicular to the surface of the skin.
The impulse causes natural damped oscillations in the tissue, which are recorded by a three-axis digital accelerometer sensor in the device.
The raw oscillation signal is then processed, and the stiffness parameter is calculated (Agyapong-Badu et al., 2016).
The MyotonPRO has previously been reported to be a reliable and valid tool for the measurement of in-vivo tissue stiffness properties (Chuang et al., 2013; Pruyn et al., 2016; Nair et al., 2014), and has demonstrated good internal consistency (coefficient of variation \textless{} 1.4\%) over sets of 10 repetitions (Aird et al., 2012).

\hypertarget{postural-stability}{%
\subsubsection{Postural stability}\label{postural-stability}}

Postural stability was assessed with a modified version of the balance error scoring system (mBESS) based on the protocol recommended by Hunt et al. (2009).
In total, each trial of the mBESS was performed without shoes (McCrory et al., 2013) and included six stances in the following order; dominant leg (DL; standing on the dominant foot with the non-dominant foot at approximately 30-degrees of hip flexion and 45-degrees of knee flexion),
non-dominant leg (NDL; standing on the non-dominant foot with the dominant foot at approximately 30-degrees of hip flexion and 45-degrees of knee flexion)
and tandem leg stance (TS; standing heel-to-toe with the non-dominant foot behind the dominant) on firm and foam (Alcan airex AG, Sins, Switzerland) surfaces respectively (Fig \ref{fig:fig2}).
To determine leg dominance, participants were asked their preferred leg to kick a ball to a target, and the chosen limb was labelled as dominant (cf.~Cingel et al., 2017).
Participants were asked to maintain each stance for a total of 20-seconds.
Participants hands were placed on hips at the level of the iliac crests.
A Sony DSC-RX10 video camera (Sony Europe Limited, Surrey, United Kingdom) was used to record each participants performance during the mBESS.

The error identification criteria from the original BESS protocol was used by the lead researcher who scored all the BESS trials.
One error was recorded if any of the following movements were observed during each trial: a) lifting hands off iliac crests;
b) opening eyes;
c) stepping, stumbling, or falling;
d) moving the thigh into more than 30 degrees of flexion or abduction;
e) lifting the forefoot or heel; and
f) remaining out of the testing position for more than 5-seconds (Riemann et al., 1999).
A maximum score of 10 errors was possible for each stance. Multiple errors occurring simultaneously were recorded as one error.
A participant was given the maximum score of 10 if they remained out of the stance position for more than 5-seconds.
A total score was calculated by summing the total number of errors recorded on all stances (DL, NLD and TS, on foam and firm surfaces).
To assess the intra-rater reliability, a single measurement, absolute agreement, two-way mixed effects model for the intraclass correlation (ICC; Koo and Li, 2016) was used on a sample of 40 participants from the first time point.
The test-retest scoring of BESS resulted in a ``good'' to ``excellent'' ICC score (ICC = 0.93, 95\% confidence interval = 0.88 - 0.96), indicating the scoring was reliable.

\hypertarget{procedure}{%
\subsection{Procedure}\label{procedure}}

At the start of the academic year (September), coaches of sports teams at a British University and local sports clubs were contacted and informed about the study. With the coaches' permission, the lead researcher attended training sessions to inform athletes about the overall purpose of the study and the requirements of participation. Athletes who met the participation criteria and volunteered to take part in the study were invited to attend scheduled testing sessions. A repeated measures prospective cohort design was used to assess athletes' major life events, stress-related physiological markers and injury status over two consecutive 12-month periods. Each participant was asked to attend a total of three data collections over a 12-month period, with each data collection separated by a four-month interval (Fig \ref{fig:fig3}). Participants provided informed consent before data collection commenced.

For the first three data collections (T1, T2 and T3), participants followed the same protocol in a specific order (Fig \ref{fig:fig4}).
To ensure all measures could be collected within a viable time-frame, participants were separated into two groups.
The first group completed all computer-based measures followed by all physical measurements, whereas the second group completed all physical measurements followed by computer-based measures.
Participants were randomly assigned to one of the two groups and remained in those groups across all time points.

\hypertarget{questionnaires}{%
\subsubsection{Questionnaires}\label{questionnaires}}

The questionnaires, which included demographic information, the LESCA, RST-PQ (T1, T2, T3) and injury status (T2, T3, T4) were completed on-line (SurveyMonkey Inc., USA, www.surveymonkey.com).
The instructions for the LESCA were modified at T2 and T3 so that participants reported major life events that had occurred since the previous testing session.
For injury reporting, participants were asked to record any injuries that they had sustained since the last data collection.
The data were downloaded from surverymonkey.com and imported into R (R Core Team, 2019) for analysis.

\hypertarget{hrv}{%
\subsubsection{HRV}\label{hrv}}

To minimise potential distractions, participants were directed to a designated quiet area in the laboratory where IBI data were recorded.
Participants were instructed to turn off their mobile devices to avoid any interference with the Bluetooth sensor.
Each chest strap was dampened with water and adjusted so it fitted tightly but comfortably, as outlined by Polar's guidelines.
Participants were seated and asked to remain as still as possible for the duration of the recording.
No attempt was made to control participants' respiratory frequency or tidal volume (Denver et al., 2007).
Inter-beat interval (IBI) data was collected for 10-minutes at a sampling frequency of 1000 Hz.

Raw, unfiltered IBI recordings were exported from the Polar Flow web service as a space delimited .txt file and imported into R (R Core Team, 2019) where the \emph{RHRV} package (Rodriguez-Linares et al., 2019) was used to calculate HRV indices.
Raw IBI data was filtered using an adaptive threshold filter, and the first 3-minutes and last 2-minutes of each recording were discarded, leaving a 5-minutes window that was used to calculate the root mean square of successive differences (RMSSD) in RR intervals following the recommendations for short term IBI recordings (Laborde et al., 2017; Malik et al., 1996). RMSSD was calculated as:

\begin{equation} 
  \overline{RR} = \frac{1}{N} \sum_{i=1}^{n} RR_i
\end{equation}

Where N is the length of the time series, and \(RR_i\) the RR interval between beats \(i\) and \(i-1\), where each beat position corresponds to the beat detection instant.

\hypertarget{muscle-stiffness}{%
\subsubsection{Muscle stiffness}\label{muscle-stiffness}}

To assess muscle stiffness, participants lay horizontally on a massage bed and four testing sites were identified on each lower limb.
The muscle belly of the rectus femoris (RF), biceps femoris (BF), medial gastrocnemius (MG) and lateral gastrocnemius (LG) sites were identified using a visual-palpatory technique to determine the exact location of each site (Chuang et al., 2012).
The visual-palpatory technique required the participant to contract the target muscle to aid the lead researcher to visually identify the muscle.
The participant was then asked to relax the muscle and the muscle was palpated to locate the muscle belly. A skin safe pen (Viscot all skin marker pen, Viscot Medical LLC, NJ) was used to mark the testing site in the centre of the muscle belly.

After the eight testing sites had been identified, the testing end of the MyotonPRO (diameter = 3 mm) was positioned perpendicular to the skin on the testing site.
A constant pre-load of 0.18 N was applied for initial compression of subcutaneous tissues.
The device was programmed to deliver five consecutive impulses, separated by a one-second interval (Morgan et al., 2018).
For each impulse, the device computed passive stiffness values, with the median of the five values being saved by the device for further analysis. In accordance with Myoton.com, a set of five measurements with a coefficient of variation (CV) of less than 3\% was accepted. Sets of measurements above 3\% were measured again to ensure data reliability.
Measurements were uploaded using MyotonPRO software and imported in R (R Core Team, 2019) for further analysis. For each participant, the sum of all eight testing sites was calculated to provide a total lower extremity stiffness score and was used for further analysis

\hypertarget{postural-stability-1}{%
\subsubsection{Postural stability}\label{postural-stability-1}}

Instructions for the mBESS were read to each participant and a demonstration of the positions was provided by the research assistant.
For each position, participants were instructed to close their eyes, rest their hands on their iliac crests and remain as still as possible for 20-seconds.
Participants were instructed to return to the testing position as quickly as possible if they lost their balance.
The video recording was started prior to the first stance position and stopped after all stances had been completed.
Each completed mBESS protocol took approximately four-minutes.
Only one trial was performed to avoid familiarisation effects across the repeated measurement (cf.~Valovich et al., 2003).
The video recordings for each participant were imported from the recording equipment (Sony DSC-RX10) and the lead researcher scored each trial using the error identification criteria.

\hypertarget{data-analysis}{%
\subsection{Data Analysis}\label{data-analysis}}

Two Bayesian Networks (BN) were used to explore the relationships between the psychological measures, physiological markers of stress and sports injury.
A BN is a graphical representation of a joint probability distribution among a set of random variables, and provides a statistical model describing the dependencies and conditional independences from empirical data in a visually appealing way (Scutari and Denis, 2014).
A BN consists of arcs and nodes that together are formally known as a directed acyclic graph (DAG), where a node is termed a parent of a child if there is an arc directed from the former to the latter (Fig \ref{fig:fig5}; Pearl, 1988).
However, the direction of the arc does not necessarily imply causation, and the relationship between variables are often described as probabilistic instead of casual (Scutari and Denis, 2014).
The information within a node can be either continuous or discrete, and a complete network can contain both continuous and discrete nodes; however, discrete networks are the most commonly used form of BN (Chen and Pollino, 2012). In discrete networks, conditional probabilities for each child node are allocated for each combination of the possible states in their parent nodes and can be used to assess the strength of a dependency in the network.

In order to use discrete networks, continuous variables must first be split into categorical levels.
When there are a large number of variables in the network, limiting the number of levels has the benefit of producing a network that is more parsimonious in terms of parameters. For example, a network with 10 variables each with two levels has 100 (10\^{}2) possible parameter combinations, however the same network with three levels has 1000 (10\^{}3) possible parameter combinations, the latter being significantly more computationally expensive. Using a larger number of splits in the data also comes at a cost of reducing the statistical power in detecting probabilistic associations, and reduces the precision of parameter estimates for the probabilistic associations that are detected because it reduces the sample-size-to-parameters ratio (Scutari and Denis, 2014). Typically, no more than three levels have been used in Bayesian networks in the sports injury literature (Olmedilla et al., 2018; Ruiz-Pérez et al., 2019)

Learning the structure of the network is an important step in BN modelling. The structure of a network can be constructed using expert knowledge and/or data-driven algorithm techniques (e.g., search and score, such as hill climbing and gradient descent algorithms; Scutari \& Denis, 2014).
The learned structure can then be used for inference by querying the network\footnote{The term ``query'' in relation to Bayesian Networks stems from Pearl's expert systems theory (1988). A query can be submitted to an expert (in this case, the network is the expert) to get an opinion, the expert then updates the querier's beliefs accordingly. Widely used texts on Bayesian Network analysis (Koller and Friedman, 2009) have widely adopted the terminology in favour of that used in traditional statistics.} and obtaining the posterior probabilities of a particular node for a given query.
The posterior distribution can be obtained by \(Pr(X|E,B) = Pr(X|E,G,\Theta)\), where the learned network \(B\) with structure \(G\) and parameters \(\Theta\), are investigated with new evidence \(E\) using the information in \(B\) (Scutari and Denis, 2014).
In the example network presented in Fig \ref{fig:fig5}, new values assigned to each of the parent nodes (e.g., both set to ``Low'') could be used to investigate what effect the new information has on the state of the child node (conditional probability of a particular state of the child node).
In a more complex network containing many nodes, the outcome of a particular node can be assessed conditional on the states of any subset of nodes in the network.
BNs therefore provide a unique and versatile approach to modelling a set of variables to uncover dependency structures within the data.

BNs have recently been used in the sport psychology literature (Olmedilla et al., 2018; Fuster-Parra et al., 2017) and offer several benefits over traditional statistical analysis.
For example, predictions can be made about any variable in the network, rather than there being a distinction between dependent and independent variables in the data, such as in linear regression models that are often used within the sport psychology literature (Olmedilla et al., 2018; Bittencourt et al., 2016).
Furthermore, the structure of a network can be obtained from both empirical data \emph{and} prior knowledge about the area of study; the latter being particularly useful when there are a large number of variables in the network, or only a small number of observations are available in the data (Xiao-xuan et al., 2008).
In such instances, a purely data driven approach to learning the network would be time-consuming due to the large parameter space, and inefficiency at identifying an approximation of the true network structure.
Prior knowledge about dependencies between variables can therefore be included in the network structure, while still allowing a data driven approach for unknown dependencies, to improve the overall computation of the network structure (Heckerman et al., 1995; Xu et al., 2015).
The following sections detail the steps taken in the current study to firstly prepare the data for each network, and then obtain the structure of each network that was used for further inference.

\hypertarget{first-network}{%
\subsubsection{First network}\label{first-network}}

\hypertarget{data-preparation}{%
\paragraph{Data Preparation}\label{data-preparation}}

Of the 351 participants that were initially recruited for the study, 94 only completed the first time point, and therefore had to be removed from the study as no injury information was obtained for these participants following the first time point.
To prepare the data for the BN, missing values in the dataset were first imputed.
Out of the 650 total measurements across all time points in the current study, there were 31 (4.64\%) missing muscle stiffness measurements and 70 (10.48\%) missing heart rate recordings. The missing data were due to technical faults in the data collection equipment and were considered to be missing completely at random.
A missing rate of 15-20\% has been reported to be common in psychological studies, and several techniques are available to handle missing values (Enders, 2003; Lang et al., 2013).
In the current study, the \emph{caret} package (Kuhn et al., 2008) was used to impute the missing values.
A bagged tree model using all of the non-missing data was first generated and then used to predict each missing value in the dataset.
The bagged tree method is a reliable and accurate method for imputing missing values in data and is superior to other commonly used methods such a median imputation (Kuhn et al., 2008).

A median split technique was used to discretise the data used in the network into ``Low'' and ``High'' levels.
All variables apart from negative and total life events were approximately normally distributed (based on visual inspection, see supplementary material Sx) and required no further transformation prior to the median split.
For the LESCA questionnaire data, a cumulative total of the current, and previous time points was calculated at each time point to account for the potential continuing effect of the life events experienced by athletes over time.
Given the limited support for a relationship between positive life events and injury (Williams and Andersen, 2007), only negative and total life events were included in the network.
Cumulative negative, and cumulative total life event scores at each time point were first log scaled so distributions were approximately normal, and then binarised using the median at each time point (nlelg and tlelg respectively).
In addition to the log scaled cumulative values, an untransformed NLE score from the first time point was included as an additional variable based on previous literature that indicates this variable should have a strong relationship with injury outcome (Ivarsson et al., 2017).

\hypertarget{network-structure}{%
\paragraph{Network structure}\label{network-structure}}

To obtain the network structure, several steps were taken to ensure that both a theoretically realistic network, and a network that was an appropriate fit to the collected data, was used for inference.
Prior knowledge about the network structure was included by providing a list of arcs that are always \emph{restricted} from being in the network (blacklist), and a list of arcs that are always \emph{included} in the network (whitelist).
Additionally, there are several scoring functions such as Bayesian Information Criteria (BIC) and Bayesian Dirichlet equivalent uniform (BDeu) that can be used to compare network structures with certain nodes and arcs included or excluded (Scutari and Denis, 2014).
To account for the repeated measures design employed in this study and to maximise the use of the data, pairs of complete cases (e.g., participants who completed T1 + T2, and T2 + T3) were used in a two-time Bayesian network (2TBN) structure (Murphy, 2002).
In the 2TBN, variables measured T2 could depend on variables measured at T1 (e.g., T1 \(\rightarrow\) T2) and variables measured at T3 could depend on variables measured at T2 (e.g., T2 \(\rightarrow\) T3).
However, arcs were blacklisted between T2 \(\rightarrow\) T1 and T3 \(\rightarrow\) T2 to preserve the order in which data was collected.
Variables were separated into two groups; ``explanatory'', for variables that did not change during the study (e.g., gender), or ``independent'', for variables that were measured at each time point and could vary during the study.
Independent variable names were suffixed with \_1 for time point T, and \_2 for time point T+1 (e.g., T1\_1 \(\rightarrow\) T2\_2 and T2\_1 \(\rightarrow\) T3\_2).
Formatting the data in this way meant participants who completed T1 and T2, but did not complete T3, could still be included in the analysis.
Table \ref{tab:table2} provides an example of the formatted data and demonstrates that participants 1 and 3 have complete data, and therefore have two rows of data each representing variables from T1 \(\rightarrow\) T2 and T2 \(\rightarrow\) T3, respectively.
Participant 2 did not complete the final data collection at T3 and therefore only has one row of data representing the variables collected at T1 and T2.
In addition to the blacklisted arcs between T2 \(\rightarrow\) T1 and T3 \(\rightarrow\) T2, the direction of arcs was restricted between independent variables and explanatory variables (e.g., independent \(\rightarrow\) explanatory); however, arcs were not restricted between explanatory \(\rightarrow\) independent variables.
Finally, arc direction was restricted between specific nodes within the explanatory variables.
Arcs from clevel \(\rightarrow\) gender, nlebase \(\rightarrow\) gender and nlebase \(\rightarrow\) ind\_team were included in the blacklist, as arcs in these directions did not make logical sense.
All subsequent models used the same blacklist.

\begin{table}[H]

\caption{\label{tab:table2}Example of the data arrangement used for the network.}
\centering
\begin{tabular}[t]{l|l|l}
\hline
\textbf{Participant} & \textbf{X\_1} & \textbf{X\_2}\\
\hline
1 & T1 & T2\\
\hline
1 & T2 & T3\\
\hline
2 & T1 & T2\\
\hline
3 & T1 & T2\\
\hline
3 & T2 & T3\\
\hline
\end{tabular}
\end{table}

\hypertarget{preliminary-network-structures}{%
\paragraph{Preliminary network structures}\label{preliminary-network-structures}}

Prior to the final network structure presented in the results section, several structures with different combinations of variables were explored. Networks were learned using a Tabu search algorithm (Russell and Norvig, 2009) and BIC was used to compare different models. A higher BIC value indicates the structure of a DAG is a better fit to the observed data (Scutari and Denis, 2014). BIC values for each combination of variables of interest are reported as the combination of variables with the highest BIC value, followed by the relative scores of the other variables in the model.

Initially, both negative life events and total life events were included in the network structure, however, the network score was improved when only nlelg or tlelg was included (highest BIC value = nleleg, BIC values relative to nlelg; tlelg only = -83.64, tlelg and nlelg = -218.31).
Additionally, despite strong evidence in the literature that both negative (NLE) and total life event (TLE) stress are related to injury occurrence (Williams and Andersen, 2007), network structures learned using the Tabu search algorithm failed to identify a relationship between NLE and injury or TLE and injury in the data.
Given that nlelg provided the highest network score, and there is a stronger relationship between negative life events and injury in the literature, an arc was whitelisted between nlelg\_1 and injured\_1 and nlelg\_2 and injured\_2 in the final network structure. TLE score was not included in the final structure.

The subscales representing the BAS (RR, RI, GDP and I) showed limited connection to other variables in the network.
Therefore, several models were run with each scale individually to find the scale that resulted in the highest BIC value (values are shown relative to the highest value).
RI provided the highest BIC value, compared to RR (-9.07), GDP (-15.06) and I (-16.33).
Including all the variables (RR, RI, GDP and I) resulted in a significantly lower score -894.33) indicating that including all the variables was not beneficial to the model structure and did not offset the cost of the additional parameters. Therefore, only RI was included in the final structure.

Finally, both total score and asymmetry for balance were included in the initial network. However, visual inspection of the network revealed no arcs between bal\_asym\_1 or bal\_asym\_2 and any other node in the network.
Therefore, balance asymmetry was removed from the final network structure.
To summarise, Table \ref{tab:table3} reports the variables that were included in the final network structure.

\begin{table}[H]

\caption{\label{tab:table3}Variables included in the final Bayesian network structure.}
\centering
\begin{tabular}[t]{l|>{\raggedright\arraybackslash}p{5.2cm}|c|c}
\hline
Variable & Definition & State 1 & State 2\\
\hline
clevel & Current competitive level & Club\_university\_county & National\_international\\
\hline
gender & Gender of the participant & Female & Male\\
\hline
hours & Number of hours spent training per week & 0-9 (Low) & >9-35 (High)\\
\hline
ind\_team & Participate in an individual or team based sport & Individual & Team\\
\hline
pi & Previous injury - Whether an injury had been sustained in the previous 12 months prior to the study & No Injury & Injury\\
\hline
nlebase & Untransformed NLE at TP 1 & 0-13 (Low) & >13-93 (High)\\
\hline
FFFS & Fight-Flight-Freeze System & 8-16 (Low) & >16-30 (High)\\
\hline
BIS & Behavioural Inhibition System & 17-38 (Low) & >38-68 (High)\\
\hline
RI & Reward Interest & 4-10 (Low) & >10-16 (High)\\
\hline
stiffness & Sum of all stiffness locations & 1543-2330 (Low) & >2330-4518 (High)\\
\hline
HRV & Root mean squared difference of successive RR intervals & 2.03-4 (Low) & >4-5.94 (High)\\
\hline
balance & Total balance score & 5-15 (Low) & >15-46 (High)\\
\hline
nlelg\_1 & Log NLE at TP 1 & 0-2.64 (Low) & >2.64-4.54 (High)\\
\hline
nlelg\_2 & Log NLE at TP 2 & 0-3.04 (Low) & >3.04-5.19 (High)\\
\hline
nlelg\_3 & Log NLE at TP 3 & 0-3.18 (Low) & >3.18-4.79 (High)\\
\hline
\end{tabular}
\end{table}

Preliminary network structures also revealed strong dependencies between the same variables at sequential time points. For example, the probability that stiffness\_1 and stiffness\_2 were both ``High'', or both ``Low'' was approximately 80\%. Including the arcs between the same variables from X\_1 → X\_2 did not provide any theoretically meaningful information to the network structure as the majority of participants would either be in a ``Low'' or ``High'' state for each pair of variables in the network. Therefore, these arcs were blacklisted from the network.
To obtain the final network, the appropriate blacklist and whitelists were provided and a Tabu search algorithm identified the remaining structure of the network.
The final network structure was obtained by averaging 1000 bootstrapped models (Efron and Tibshirani, 1994) to reduce the impact of locally optimal, but globally suboptimal network learning, and to obtain a more robust model (Olmedilla et al., 2018). Arcs that were present in at least 30\% of the models were included in the averaged model.
The strength of each arc was determined by the percentage of models that the arc was included in, independent of the arc's direction. An arc strength of 1 indicated that the arc was always present in the network, with the value decreasing as arcs were found in fewer networks. In the respective study arcs above 0.5 were considered ``significant'' with arcs below 0.5 and above 0.3 ``non-significant'' (Scutari and Nagarajan, 2013).
Arcs below 0.3 were not included in the model. The full table of arc strengths for the first and second network are available in S2 Table and S3 Table respectively.

\hypertarget{network-inference}{%
\paragraph{Network Inference}\label{network-inference}}

Conditional probability queries (CPQ) were used to perform inference on both network structures. To conduct a CPQ, the joint probability distribution of the nodes was modified to include a new piece of evidence. The query allows the odds of a particular node state (e.g., injured\_1 = ``injured'') to be calculated based on the new evidence. CPQs were performed using a likelihood weighting approach; a form of importance sampling where random observations are generated from the probability distribution in such a way that all observations match the evidence given in the query. The algorithm then re-weights each observation based on the evidence when computing the conditional probability for the query (Scutari and Denis, 2014). Inference was first performed on arcs that had a strength greater than 0.5 between the explanatory variables and independent variables and between different independent variables in the network. Of particular interest were the variables that were connected to ``injured'' nodes, which were examined in the network using the Markov blanket of ``injured\_1'' and ``injured\_2''. A Markov blanket contains all the nodes that make the node of interest conditionally independent from the rest of the network (Fuster-Parra et al., 2017). CPQ's were used to determine what effect the variables in the Markov blanket of injured nodes had on the probability of the injured node being in the ``injured'' state.

\hypertarget{second-network}{%
\subsubsection{Second network}\label{second-network}}

\hypertarget{data-preparation-1}{%
\paragraph{Data preparation}\label{data-preparation-1}}

Change scores between timepoint 1 and timepoint 2 were standardized to allow relative changes between variables to be compared.

For the second network, change scores for continuous variables between T1 \(\rightarrow\) T2 and T2 \(\rightarrow\) T3 were standardized to allow relative changes between variables to be compared.
The ``injured'' variable was also modified to represent whether a participant had sustained an injury at any point over the duration of the study or were healthy for the duration of the study.
Setting the data up in this way enabled the construction of a network that explicitly modelled the \emph{amount} of change within variables between time points, as opposed to the first network that only captured changes when the median threshold was crossed from ``Low'' to ``High''.

\hypertarget{network-structure-1}{%
\paragraph{Network structure}\label{network-structure-1}}

Similar to the first network, blacklists were used to prevent arcs from independent variables \(\rightarrow\) explanatory variables.
In addition, ``nlebase'' variable was dropped from the list of explanatory variables to allow the \emph{changes} in negative life events to be the only life event variable in the network.
The final network was obtained using the same approach as the first network.

\hypertarget{network-inference-1}{%
\paragraph{Network Inference}\label{network-inference-1}}

Conditional probability queries (CPQ) were again used to perform inference the network structures.
The Markov blanket of the ``injured'' node was of particular interest, and the probability of injury was investigated with combinations of variables in the Markov blanket below the mean change, at the mean change and above the mean change.
Initial visual inspection of the network structure also revealed arcs between BIS \(\rightarrow\) FFFS and HRV \(\rightarrow\) FFFS.
To investigate this relationship further, random samples were generated for BIS, HRV and FFFS based on the conditional distribution of the nodes included as evidence in the query.
The samples were then examined using a Bayesian linear regression models with the \emph{brms} package (Bürkner, 2017) to determine the relationship between these nodes.
Weakly regularising priors (normal prior with mean of 0 and standard deviation of 5) were used for all parameters in the model.

\hypertarget{results}{%
\section{Results}\label{results}}

During the study, 26\% of participants reported at least one injury with an average severity of 11 \(\pm\) 31, days (range = 2 - 365 days).
Both male and female participants reported a greater number of acute compared to chronic injuries (male, acute = 85 {[}69\%{]}, chronic = 39 {[}31\%{]}; female, acute = 38 {[}72\%{]} chronic = 15 {[}28\%{]}), and non-contact injuries were more common than contact injuries (male, non-contact = 83 {[}67\%{]}, contact = 39 {[}31\%{]}; female, non-contact = 35 {[}66\%{]} contact = 18 {[}34\%{]}).
Table \ref{tab:table4} shows the number and percentage of injury types sustained by both male and female participants.

\begin{table}[H]

\caption{\label{tab:table4}The number and percentage (\%) of types of injuries sustained by male and female participants.}
\centering
\begin{tabular}[t]{l|l|l|l|l}
\hline
\multicolumn{1}{c|}{\textbf{ }} & \multicolumn{2}{c|}{\textbf{Female}} & \multicolumn{2}{c}{\textbf{Male}} \\
\cline{2-3} \cline{4-5}
\textbf{} & \textbf{Lower body} & \textbf{Upper body} & \textbf{Lower body} & \textbf{Upper body}\\
\hline
Joint / ligament & 14 (36) & 5 (36) & 37 (43) & 14 (38)\\
\hline
Muscle / tendon & 17 (44) & 6 (43) & 45 (52) & 12 (32)\\
\hline
Other & 8 (21) & 3 (21) & 5 (6) & 11 (30)\\
\hline
\multicolumn{5}{l}{\textit{Note: }}\\
\multicolumn{5}{l}{Other included bone, skin and brain injuries.}\\
\end{tabular}
\end{table}

\hypertarget{first-network-structure}{%
\subsection{First network structure}\label{first-network-structure}}

The first network structure obtained from the data is shown in Fig \ref{fig:fig6}.
Several of the explanatory variables showed strong connections with independent variables in the network.
The arc from nlebase \(\rightarrow\) RI\_1 had a strength of 0.84, and the probability of RI\_1 being in the ``High'' state increased from 0.23 to 0.47 when nlebase increased from ``Low'' to ``High''.
The ind\_team node had strong arcs to hours (0.90) and nlebase (0.84).
Individual athletes were more likely to have ``High'' hours per week (0.84) compared to team-based athletes (0.60).
Individual athletes were also more likely to have ``High'' negative life events in the 12 months preceding the start of the study compared to team based athletes (individual athletes = 0.65, team-based athletes = 0.41).
The arcs from gender \(\rightarrow\) stiffness\_1 and gender \(\rightarrow\) stiffness\_2 were 0.76, and 0.65 respectively, with males more likely to have ``High'' stiffness compared to females (males = 0.62, females = 0.43).
The arc from pi \(\rightarrow\) stiffness\_1 was 0.55 with athletes who reported an injury in the preceding 12 months more likely to have ``High'' (0.65) compared to ``Low'' (0.35) stiffness.
The arc from clevel \(\rightarrow\) balance\_1 had a strength of 0.51, with lower level performers more likely to have decreased balance ability (0.48), compared to national level athletes (0.29).
Arcs were also present between independent variables in the network. Strong arcs were present between BIS\_1 \(\rightarrow\) FFFS\_1 (0.98) and BIS\_2 \(\rightarrow\) FFFS\_2 (0.68).
In both instances, ``High'' FFFS was more likely when BIS was ``High'' (0.65 for \_1, 0.61 for \_2) compared to ``Low'' (0.32 for \_1, 0.36 for \_2).
The arc between nlelg \(\rightarrow\) BIS had a strength of 0.62 for nlelg\_1 \(\rightarrow\) BIS\_1, however no arc was present between nlelg\_2 and BIS\_2.
For nlelg\_1 \(\rightarrow\) BIS\_1, ``Low'' negative life events increased the probability of BIS being in the ``High'' state from 0.33 to 0.55.

\hypertarget{markov-blanket-for-injured_1}{%
\subsubsection{Markov blanket for injured\_1}\label{markov-blanket-for-injured_1}}

The Markov blanket for injured\_1, which contained hours spent training per week (hours), negative life events (nlelg\_1), muscle stiffness (stiffness\_1), current competitive level (clevel) and balance (balance\_1), is shown in Fig \ref{fig:fig7}.
The arc between nlelg\_1 and injured\_1 was fixed in the network, so has the maximum strength of 1.

The conditional probability query (CPQ) for injured\_1 in the ``injured'' state for all variables that were directly linked to injured\_1 is shown in Table \ref{tab:table5}.
The probability of injured\_1 = ``injured'' rose from 0.17 to 0.31 when stiffness was ``High'' compared to ``Low''.
Negative life events had a negligible effect when moving from the ``Low'' to ``High'' state.

\begin{table}[H]

\caption{\label{tab:table5}Probability of injured\_1 being in the ``injured'' state, conditional on each variable.}
\centering
\begin{tabular}[t]{l|r|r}
\hline
\textbf{Variable} & \textbf{Low} & \textbf{High}\\
\hline
balance\_1 & 0.21 & 0.30\\
\hline
hours & 0.18 & 0.28\\
\hline
nlelg\_1 & 0.24 & 0.26\\
\hline
stiffness\_1 & 0.17 & 0.31\\
\hline
\end{tabular}
\end{table}

The second CPQ investigated the outcome of injured\_1 being ``injured'' conditional on all variables in the Markov blanket.
The Markov blanket contained five nodes, each with two possible states resulting in \(2^5\) combinations of variables, therefore only the three lowest and highest probabilities are shown in Table \ref{tab:table6} (complete results in S4 Table).
The combination of lower competitive level, ``High'' hours per week, ``Low'' negative life events, ``High'' balance and ``High'' stiffness resulted in a probability of 0.53 for injured\_1 being in the ``injured'' state.
When all variables were in the ``Low'' state the probability of ``injured'' was approximately 0.04
Negative life events only had a substantial effect on injured\_1 when all other variable were fixed to ``Low''.
In this instance the probability of injured\_1 being ``injured'' rose from 0.04 to 0.19, when negative life events was in the ``Low'' and ``High'' states respectively.

\begin{table}[H]

\caption{\label{tab:table6}Highest and lowest probability of injured\_1 being in the 'injured' state, conditional on the all variables in the Markov blanket for injured\_1.}
\centering
\begin{tabular}[t]{r|l|l|l|l|l}
\hline
\textbf{Probability} & \textbf{clevel} & \textbf{hours} & \textbf{nlelg\_1} & \textbf{stiffness\_1} & \textbf{balance\_1}\\
\hline
\multicolumn{6}{l}{\textbf{Highest}}\\
\hline
\hspace{1em}0.53 & club\_university\_county & High & Low & High & High\\
\hline
\hspace{1em}0.46 & national\_international & High & Low & High & Low\\
\hline
\hspace{1em}0.43 & national\_international & High & Low & High & High\\
\hline
\multicolumn{6}{l}{\textbf{Lowest}}\\
\hline
\hspace{1em}0.06 & national\_international & Low & Low & Low & Low\\
\hline
\hspace{1em}0.05 & national\_international & Low & Low & Low & High\\
\hline
\hspace{1em}0.04 & club\_university\_county & Low & Low & Low & Low\\
\hline
\end{tabular}
\end{table}

\hypertarget{markov-blanket-of-injured_2}{%
\subsubsection{Markov blanket of injured\_2}\label{markov-blanket-of-injured_2}}

The Markov blanket for injured\_2 is shown in Fig \ref{fig:fig8} and contained gender, FFFS\_1, stiffness\_2, balance\_2 and rmssd\_2.
The arc between stiffness\_2 \(\rightarrow\) injured\_2 was comparable to the arc between stiffness\_1 \(\rightarrow\) injured\_1.
Very weak arcs (0.3) between injured\_2 \(\rightarrow\) balance\_2 and injured\_2 \(\rightarrow\) rmssd\_2 were also present in the Markov blanket for injured\_2.
Results of the first query for injured\_2 in the ``injured'' state are presented in table \ref{tab:table8}.
Similar to injured\_1, stiffness\_2 doubled the probability of injured\_2 being ``injured'' from 0.13 in the ``Low'' state to 0.27 in the ``High'' state.
FFFS\_1 in the ``Low'' state increased probability of injured\_2 being ``injured'' by 0.19 compared to the ``High'' state.
``High'' negative life events decreased the probability of injury from 0.24 to 0.19.

\begin{table}[H]

\caption{\label{tab:table8}Probability of injured\_2 being in the ``injured' state, conditional on each variable in the Markov blanket for injured\_2.}
\centering
\begin{tabular}[t]{l|l|l}
\hline
\textbf{Variable} & \textbf{Low} & \textbf{High}\\
\hline
balance\_2 & 0.17 & 0.27\\
\hline
FFFS\_1 & 0.3 & 0.11\\
\hline
nlelg\_2 & 0.24 & 0.19\\
\hline
rmssd\_2 & 0.24 & 0.19\\
\hline
stiffness\_2 & 0.13 & 0.27\\
\hline
\end{tabular}
\end{table}

The conditional probabilities based on all the variables in injured\_2 Markov blanket are presented in Table \ref{tab:table9}.
Again, only the three lowest and highest probabilities are shown (complete results in S5 Table).
The combination ``Low'' FFFS\_1, ``High'' stiffness\_2, ``High'' balance resulted in the greatest probability of injured\_2 being ``injured'', with the highest probability of injury being 0.52.
With all other variables held in the ``High'' state, the probability of injured\_2 being ``injured'' rose from 0.14 to 0.34 when FFFS\_1 was in the ``Low'' compared to ``High'' state.
The combination of ``Low'' stiffness, ``Low'' balance and ``High'' FFFS resulted in the lowest probability of injured\_2 being ``injured''.

\begin{table}[H]

\caption{\label{tab:table9}Highest and lowest probability of injured\_2 being in the 'injured' state, conditional on the all variables in the Markov blanket for injured\_2.}
\centering
\begin{tabular}[t]{l|l|l|l|l|l}
\hline
\textbf{Probability} & \textbf{FFFS\_1} & \textbf{nlelg\_2} & \textbf{stiffness\_2} & \textbf{rmssd\_2} & \textbf{balance\_2}\\
\hline
\multicolumn{6}{l}{\textbf{Highest}}\\
\hline
\hspace{1em}0.52 & Low & Low & High & Low & High\\
\hline
\hspace{1em}0.45 & Low & High & High & Low & High\\
\hline
\hspace{1em}0.43 & Low & Low & High & High & High\\
\hline
\multicolumn{6}{l}{\textbf{Lowest}}\\
\hline
\hspace{1em}0.06 & High & High & Low & Low & Low\\
\hline
\hspace{1em}0.05 & High & Low & Low & High & Low\\
\hline
\hspace{1em}0.04 & High & High & Low & High & Low\\
\hline
\end{tabular}
\end{table}

\hypertarget{second-network-structure---changes-within-variables}{%
\subsection{Second network structure - changes within variables}\label{second-network-structure---changes-within-variables}}

The network for changes within variables is presented in Fig \ref{fig:fig9}.
An arc between BIS \(\rightarrow\) FFFS with strength 1 was present in the network.
Arcs between clevel \(\rightarrow\) BIS and gender \(\rightarrow\) stiffness had a strength of 0.60.
The arc between RMSSD \(\rightarrow\) FFFS was 0.56.
The arcs between BIS \(\rightarrow\) FFFS and RMSSD \(\rightarrow\) FFFS were examined further by drawing random observations from the conditional probability distribution and examining the relationship in a Bayesian linear regression model.
The use of a separate linear regression enabled the interaction between BIS and RMSSD to be examined.

Results from the Bayesian linear regression model are presented in Table \ref{tab:table10} and include 95\% credible intervals (CrI).
Increases in BIS were associated with increases in FFFS (\(b\) = -0.14, 95\% CrI = {[}-0.20, -0.08{]}), whereas positive changes in RMSSD where associated with decreased changes in FFFS (\(b\) = 0.4, 95\% CrI = {[}0.34, 0.45{]}).
There was no clear interaction between RMSSD and BIS (\(b\) = -0.04, 95\% CrI = {[}-0.10, 0.02{]}).

\begin{table}[H]

\caption{\label{tab:table10}Estimate, error and 95\% credible intervals for the fixed effects in the linear model containing FFFS, BIS and RMSSD.}
\centering
\begin{tabular}[t]{l|r|r|l}
\hline
\textbf{Term} & \textbf{Estimate} & \textbf{Error} & \textbf{95\% CI}\\
\hline
Intercept & 0.00 & 0.03 & [-0.06, 0.05]\\
\hline
BIS & 0.40 & 0.03 & [0.34, 0.45]\\
\hline
rmssd & -0.14 & 0.03 & [-0.20, -0.08]\\
\hline
BIS:rmssd & -0.04 & 0.03 & [-0.10, 0.02]\\
\hline
\end{tabular}
\end{table}

The Markov blanket for the ``injured'' node contained previous injury, gender, hours per week, stiffness and nlec (Fig \ref{fig:fig10}).
For stiffness and nlec the values in the nodes represent the standardised change between time point.
Combinations of nlec and stiffness at one SD below the mean change, at the mean change, and 1 SD above the mean change are presented in table \ref{tab:table11}.
Changes in both nlec and stiffness 1SD above the mean change resulted in a probability of being injured of 0.71 over the duration of the study.
With stiffness held at the mean change, the probability of ``injured'' rose from 0.35 to 0.64 with nlec at 1 SD below an 1 SD above respectively.

\begin{table}[H]

\caption{\label{tab:table11}The probability of injury with values of stiffness and nlec held at 1SD below the mean change, at the mean change and 1 SD above the mean change.}
\centering
\begin{tabular}[t]{r|l|l}
\hline
\textbf{Probability} & \textbf{nlec} & \textbf{Stiffness}\\
\hline
0.71 & +1SD & +1SD\\
\hline
0.64 & +1SD & mean\\
\hline
0.62 & +1SD & -1SD\\
\hline
0.52 & mean & +1SD\\
\hline
0.44 & mean & mean\\
\hline
0.43 & mean & -1SD\\
\hline
0.42 & -1SD & +1SD\\
\hline
0.35 & -1SD & mean\\
\hline
0.35 & -1SD & -1SD\\
\hline
\end{tabular}
\end{table}

Table \ref{tab:table12} shows the three highest and lowest probabilities for injury for all variables in the Markov blanket.
The combination of 1 SD above the mean change for nlec and stiffness and ``High'' hours per week and previous injury resulted in the highest probability that an injury would be sustained during the study (0.77).
In contrast, below average changes in nlec and stiffness combined with ``Low'' hours per week and no previous injury resulted in the lowest probability of an injury (0.12).

\begin{table}[H]

\caption{\label{tab:table12}Highest and lowest probability of injury, conditional on the all variables in the Markov blanket for ``injured''.}
\centering
\begin{tabular}[t]{r|l|l|l|l}
\hline
\textbf{Probability} & \textbf{hours} & \textbf{pi} & \textbf{nle} & \textbf{stiffness}\\
\hline
\multicolumn{5}{l}{\textbf{Highest}}\\
\hline
\hspace{1em}0.77 & High & injury & +1SD & +1SD\\
\hline
\hspace{1em}0.74 & High & no injury & +1SD & +1SD\\
\hline
\hspace{1em}0.72 & Low & injury & +1SD & +1SD\\
\hline
\multicolumn{5}{l}{\textbf{Lowest}}\\
\hline
\hspace{1em}0.15 & Low & no injury & -1SD & +1SD\\
\hline
\hspace{1em}0.13 & Low & no injury & -1SD & mean\\
\hline
\hspace{1em}0.11 & Low & no injury & -1SD & -1SD\\
\hline
\end{tabular}
\end{table}

\hypertarget{discussion}{%
\section{Discussion}\label{discussion}}

\hypertarget{summary-fo-results}{%
\subsection{Summary fo results}\label{summary-fo-results}}

(not sure where this would go /if needed at all?). The first BN revealed both high stiffness and poor balance resulted in an increased probability of injury. However, contrary to previous literature, negative life event stress had a negligible effect on the probability of injury.
In the second network that modelled changes between time points in the study, increases in muscle stiffness was found to increase the risk of injury. In addition, the probability of injury was further increased when there were increases in NLE stress.

(Main discussion starts here..)

Informed by Appaneal and Perna's (2014) extension to the widely cited Williams and Anderson's (1998) stress-injury model, the purpose of this study was to examine how psychosocial stress-related factors, and physiological stress-related markers may interact and act synergistically to increase the risk of injury; therein addressing several of the limitations of the sport injury literature (Johnson et al., 2014).
Potential relationships between psychosocial stress-related factors, physiological stress-related markers of stress and injury were investigated using two BN structures; the first was a two-time Bayesian Network that investigated the relationships between variables \emph{across} time points in the study (Fig \ref{fig:fig5}), and the second network used differential equations to model the changes in variables \emph{between} time points (Fig \ref{fig:fig8}).

The first network revealed several links between the injured nodes and other variables in the network.
For example, Fig \ref{fig:fig6} and Fig \ref{fig:fig7} show the Markov blankets for the injured\_1 and injured\_2 nodes in the first network and include all the variables that had a direct effect on the probability of injury.
The combination of high stiffness and poor balance resulted in the highest probability of injury in the Markov blankets for ``injured'' nodes.
The presence of these factors at both injured nodes indicates that the combined action of these variables is important for determining an athlete's risk of injury.
In the second network, the highest probability of injury was observed when changes in stiffness \emph{and} negative life events were both greater than average (Table \ref{tab:table11}), indicating that the combination of changes in psychological and physiological measures of stress may combine additively to increase the risk of injury (Appaneal and Perna, 2014).

Of all the variables measured in the study, muscle stiffness appeared to be most strongly related to injury.
Both ``High'' levels of stiffness in the first network, and greater than average increases in stiffness in the second network were found to increase the risk of injury.
In this study, a novel hand-held device (MyotonPRO) was used to measure muscle stiffness.
To date, one of only a small number of studies that have used the MyotonPRO to explore the relationship between muscle stiffness and sports injury, found that increased muscle stiffness in the soleus and Achilles tendon was related to increased injury incidence in elite level netball players (Pickering-Rodriguez et al., 2017).
The results of the current study build on these findings, with a larger sample of athletes from a range of different sports, strengthening the evidence for a relationship between higher levels of muscle stiffness and injury.
However, high levels of muscle stiffness, as measured by the MyotonPRO, have also been found to be related to improved performance, with elite level athletes having increased lower extremity stiffness (Pruyn et al., 2015; Kalkhoven and Watsford, 2017).
Collectively, therefore, these findings suggest that while muscle stiffness plays a vital role in performance, increased levels of stiffness also increase the risk of injury, with each athlete likely to have an optimum level of stiffness that maximises performance while minimising the risk of injury (Butler et al., 2003).
Additionally, high stiffness may only increase the risk of injury if other factors are also present.
To elaborate, the combination of high stiffness \emph{and} poor balance was found to result in the greatest probability of injury.
In contrast, athletes with high stiffness and good balance were less likely to be injured, suggesting that improved postural stability may counteract the potential harmful effects of high levels of muscle stiffness.
Several studies have identified how balance (Romero-Franco et al., 2014; Trojian and McKeag, 2006) and muscle stiffness (Butler et al., 2003; Pickering-Rodriguez et al., 2017) are related to injury individually; however this study has demonstrated how these two factors may have an additive effect in relation to injury occurrence.

In addition to stiffness, balance is also linked to injury at both injured nodes in the first network, however the strength of the arc was only 0.35 and 0.30 from balance \(\rightarrow\) injured\_1 and balance \(\rightarrow\) injured\_2 respectively.
Despite the weak arc strength, a ``High'' balance score, indicating impaired postural stability, was found to increase the probability of injury.
This finding is consistent with previous research that has reported an association between decrements in postural stability and increased injury risk (Riemann et al., 1999; Romero-Franco et al., 2014; Trojian and McKeag, 2006).
Postural stability is often used as an indicator of athlete performance level, with higher level athletes demonstrating better postural stability over their lower level counterparts (Paillard et al., 2006).
In the current study, athletes who competed at a higher level were also more likely to have good balance (``Low'' balance), compared to their lower level counterparts.
These findings suggest that better postural stability is associated with both a higher level of performance and a lower probability of sustaining an injury, reinforcing the importance of postural stability as a feature of athletic training programmes designed to prepare athletes for the demands of high intensity training and competition (Hrysomallis, 2011).

Of the psychological variables in the study, negative life events have previously been reported to be most strongly associated with injury (Williams and Andersen, 2007; Ivarsson et al., 2017).
In the current study, the second network revealed that greater than average increases in negative life event stress increased the probability of being injured during the study period.
However, negative life event stress had almost no effect on the probability of injury in the first network.
This finding suggests that the \emph{relative change} in life events may be more important than the \emph{absolute score} for life events, despite the latter being commonly used in sports injury research to date.
For example, an athlete who reports a negative life event score of 1 during the first time point, but then a score of 5 at the second time point will have a 400\% increase in their life event score.
Although the absolute score would be ``Low'', the relative increase could have been caused by a significant event in the athlete's life, that could have a considerable psychological and physiological effect (Appaneal and Perna, 2014).
Future research should therefore consider study designs and appropriate analysis methods that enable relative changes in an individual athlete's life events to be assessed (cf.~Ivarsson et al., 2014).

The finding that negative life events had almost no impact on the probability of injury in the first network was surprising given that the majority of research has consistently identified major life events, particularly those events with a \emph{negative} valence, as the strongest predictor of injury in Williams and Andersen's (Williams and Andersen, 1998) model (Ivarsson et al., 2017).
During the initial network structure development, no arcs between the negative life event nodes and injured nodes were found by the Tabu search algorithm.
However, given the strength of the literature indicating that negative life events are related to injury, an arc was fixed between nlelg\_1 \(\rightarrow\) injured\_1 and nlelg\_2 \(\rightarrow\) injured\_2 to allow this relationship to be examined more closely.
When negative life events were ``High'' the probability of injury showed a negligible change at the injured\_1 node and decreased by -0.05 at the injured\_2 node.
One possible explanation for these findings may be due to the use of the LESCA questionnaire in a repeated measures design.
In the original LESCA instructions, participants are asked to report major life events that have occurred over the previous 12-months (Petrie, 1992).
However, in the current study, participants completed the LESCA at approximately 4-month intervals after baseline and were therefore asked to report any events which had occurred \emph{since the previous data collection session}, to avoid inflated scores caused by reporting the same event on multiple occasions.
The reduced 4-month time interval between data collections may have reduced the likelihood for life events listed in the LESCA to have taken place.
For example, at the second and third time points, 26\% of participants reported 0 negative life events for the preceding four-month period.
Simply, it may be that the items on the LESCA are less suitable for repeated measurements with durations shorter than the original 12-months than a measure that captures minor life events (cf.~Fawkner et al., 1999).

Another possible explanation for the findings for major/negative life events is that participants in the study may have had access to the necessary coping resources to mitigate against the effects of any negative life event stress they experienced.
Williams and Andersen's (Williams and Andersen, 1998) model proposed a number of coping resources that were either directly related to injury or moderated the relationship between life stress and injury occurrence; for example, general coping strategies (e.g., good sleeping habits and self-care), social support systems and stress management skills.
Although coping was not measured in the current study, several studies have found high levels of social support can reduce the risk of injury (Petrie, 1993; Petrie et al., 2014; Johnson et al., 2014).
Therefore, future research should consider including a measure of coping alongside that of life event stress to help explain the possible moderating effect.

Of the remaining variables, both FFFS and RMSSD were also linked to injury.
A weak arc was observed between RMSSD\_2 \(\rightarrow\) injured\_2 (arc strength = 0.30), however no arc was present between RMSSD\_1 and injured\_1, suggesting the link between RMSSD and injury was not as certain as muscle stiffness and balance, where stronger arcs existed at both of the injured nodes.
Despite the uncertainty regarding the relationship between injury and RMSSD in the first network, ``Low'' RMSSD increased the probability of injury from 0.17 (RMSSD = ``High'\,') to 0.27 (RMSSD =''Low'\,').
This finding is consistent with previous research that has found reduced RMSSD to be indicative of illness or maladaptation to training due to decreased parasymapthic activity, which often precedes injury (Williams et al., 2017; Bellenger et al., 2016; Gisselman et al., 2016).
An arc between FFFS\_1 and injured\_2 (arc strength = 0.40) was also observed in the first network, where the risk of injury was increased from 0.13 to 0.29 with FFFS in the ``High'' and ``Low'' states respectively.
Interestingly, the ``Low'' FFFS score was also related to injuries at subsequent time points.
One possible explanation for this finding could be that those athletes who reported ``Low'' FFFS score were less fearful, and may therefore engage in more risk taking behaviours, increasing the probability of injury.
The RST theory proposes that higher levels of FFFS increase avoidance motivation (Corr et al., 2016), and therefore ``High'' FFFS may have acted as a deterrent from taking risks while training and competing, reducing exposure to situations that could have resulted in injury.

In the first network, ``High'' BIS was associated with ``High'' FFFS, while in the second network, increases in BIS were associated with increases in FFFS.
RST proposes that the combination of high BIS and high FFFS is likely to result in a more anxious disposition due to high levels of avoidance and high goal conflict characterised by high levels of FFFS and BIS (Corr, 2013).
High levels of anxiety and anticipation of stressful situations have been associated with reductions in HRV indices including RMSSD (Chalmers et al., 2014; Pulopulos et al., 2018), which is supported by the negative relationship between FFFS and RMSSD in the second network (Table \ref{tab:table10}).
These findings align with the proposed actions of the RST theory (Corr et al., 2016).
For example, high levels of BIS are proposed to be the result of goal conflict between the FFFS (avoidance) and BAS (approach) systems.
The goal conflict is likely to elicit a physiological response (e.g., decreased HRV) in preparation to engage in the required behaviour to resolve the goal conflict (Corr et al., 2016).
To extend these findings, the BAS should also be considered.
Specifically, to establish how the BAS and FFFS interact, and how these two systems affect the BIS.
However, in the current study, initial network structures revealed the BAS sub-scales to have limited connectivity with other measures in the network, therefore only one of the BAS sub-scales (RI) was included in the final network structure.
In the first network, RI\_1 was connected to both BIS\_1 and BIS\_2, and in both instances, the probability of ``High'' BIS was increased when RI was also ``High''.
However, the arcs between RI\_1 and BIS were weak (\textless{} 0.50), and RI represents only one component of the BAS system.
Other BAS factors such as impulsivity may be more closely related to risk-taking behaviours and may reveal additional links to sports injury.
Therefore, a more detailed examination of the different elements of RST, and specifically the BAS in relation to injury occurrence is warranted.

The current research had several strengths, including the longitudinal repeated measures design and modelling approach.
A major critique of the sport injury literature has been the use of only one wave of measurement that may not be reflective of and capture the dynamic nature of the variables that are associated with injury (Johnson et al., 2014).
The longitudinal repeated measures design of the current study allowed \emph{changes} over time and between time points to be captured and explored.
Another significant strength of the current research was the interdisciplinary approach, which enabled an examination of the complex interplay between psychosocial and physiological markers of stress.
Although there are unique and significant challenges with research employing longitudinal repeated measures designs, they provide for far more fine-grained understanding of the dynamic relationships between stress-related factors and injury occurrence in athletes.
Sport injury research has been criticised for adopting analytic approaches that are reductionist in nature (Bittencourt et al., 2016) that fail to account for the complex, emergent behaviour that is characteristic of injury occurrence.
To address this issue, Bayesian networks (BN) were used to more closely align with the complex, multifactorial nature of injury.
The networks allowed several markers of stress that were free to interact with each other, as well as injury, to be explored.
Consequently, BN's provided a contemporary approach that improved upon traditional methods such as logistic regression (Olmedilla et al., 2018).

As with most research, there were several limitations with the present study.
Firstly, the choice was made to binarise variables in the first network so only ``Low'' and ``High'' states were observed.
Although binarising variables is a common procedure in Bayesian network analysis and has several advantages, Qian and Miltner Qian and Miltner (2015) highlighted that both a loss of statistical accuracy and potential difficulty in subsequent interpretation of the model may arise when following a binarising procedure.
For example, the meaning of a ``Low'' and ``High'' value in the current study is only meaningful for the population that was studied, and there could be additional levels within each category that were not investigated.
A second limitation was the nature of the physiological measures used in the current study.
In order to collect data on a large sample of participants, suitable measures were required to ensure the viability of the data collection; however, some of these measures may not have been sensitive enough to detect more subtle variation in athletes.
For example, postural stability could have been assessed with the use of a force plate, which is considered the gold standard, to provide detailed data and enable a more fine-grained analysis (Ross et al., 2011).

In addition to the future directions already outlined, the findings from the current study offer several avenues for future research.
Although the current study used a range of measures to capture ``stress'' from both a psychological and physiological perspective, there may be additional measures available that could provide further insight into the relationship between stress-related factors and sports injury.
For example, stress hormones such as cortisol have been found to be a marker of both psychological and training-related stress (Appaneal and Perna, 2014; Perna and McDowell, 1995), and could help elucidate the relationship between stress and injury.
Additionally, although the LESCA is the most widely used measures of major life events in sports injury research, the current study found several limitations with using the LESCA in a repeated measures design, including how the items were scored.
For example, there is no way to differentiate between an athlete who has answered four items as moderately negative, and one item as extremely negative.
Both responses would be scored a ``-4''; however, there could be vastly different psychological and physiological effects between moderately negative and extremely negative events. Therefore, future research could develop a modified version of the LESCA that could distinguish between these types of responses and their effects.

In conclusion, the purpose of this research was to explore the multifaceted nature of the stress-injury relationship, and several psychosocial and physiological markers were found to combine and exacerbate the risk of injury.
Specifically, muscle stiffness and \emph{increases} in negative life event stress were identified as strong predictors of injury, while other factors including personality characteristics and postural stability were also found to contribute to the probability of injury occurrence.
Taken together, the interdisciplinary approach coupled with the advanced analytical techniques used and complex systems framework has provided a novel examination of the stress-injury relationship that has addressed many of the limitations identified in previous research.

\hypertarget{disclosureconflict-of-interest-statement}{%
\section{Disclosure/Conflict-of-Interest Statement}\label{disclosureconflict-of-interest-statement}}

The authors declare that the research was conducted in the absence of any
commercial or financial relationships that could be construed as a potential
conflict of interest.

\hypertarget{author-contributions}{%
\section{Author Contributions}\label{author-contributions}}

The statement about the authors and contributors can be up to several sentences
long, describing the tasks of individual authors referred to by their initials
and should be included at the end of the manuscript before the References
section.

\hypertarget{supplemental-data}{%
\section{Supplemental Data}\label{supplemental-data}}

Supplementary Material should be uploaded separately on submission, if there are
Supplementary Figures, please include the caption in the same file as the
figure. LaTeX Supplementary Material templates can be found in the Frontiers
LaTeX folder

\hypertarget{figures}{%
\section{Figures}\label{figures}}

\begin{figure}[!h]
\caption{{\bf Stress and injury model (Williams and Andersen, 1998).}}
\label{fig:fig1}
\end{figure}

\begin{figure}[!h]
\caption{{\bf mBESS positions (A-F).}
Top row, firm surface. Bottom row, foam surface. Left column, dominant leg stance. Middle column, non-dominant leg stance. Right column, Tandem leg stance.}
\label{fig:fig2}
\end{figure}

\begin{figure}[!h]
\caption{{\bf Study design.}
For each time point (T), each box contains the number of participants who completed the data collection (N), the measures used for data collection and the approximate date of the data collection.}
\label{fig:fig3}
\end{figure}

\begin{figure}[!h]
\caption{{\bf Session protocol.}
Outline of the protocol for each data collection.}
\label{fig:fig4}
\end{figure}

\begin{figure}[H]
\caption{{\bf Example network.}
A simple discrete network contain nodes, possible states of the nodes and the arcs connecting nodes.}
\label{fig:fig5}
\end{figure}

\begin{figure}[H]
\caption{{\bf Network structure.}
The full Bayesian network structure was plotted using the strength.plot function in bnlearn. The strength of each arc is shown graphically by the style of the arc. Thin, dashed arcs indicate the weakest arcs (arc strength below 0.50), whereas thick solid arcs indicate the strongest arcs (arc strength of 1). White nodes in the network indicate the explanatory variables, blue nodes indicate T1\_1 and T2\_1 variables, and red nodes indicated T2\_2 and T3\_2 variables. The injured\_X nodes have been coloured gold as they are the main nodes of interest within the network.}
\label{fig:fig6}
\end{figure}

\begin{figure}[H]
\caption{{\bf Markov blanket of injured\_1.}
Arc strengths are included as arc labels.}
\label{fig:fig7}
\end{figure}

\begin{figure}[H]
\caption{{\bf Markov blanket for injured\_2.} }
\label{fig:fig8}
\end{figure}

\begin{figure}[H]
\caption{{\bf Change network.}
Network structure of the changes within variables between time points.}
\label{fig:fig9}
\end{figure}

\begin{figure}[H]
\caption{{\bf Markov blanket for the injured node in the network reflecting changes within variables between time points.}}
\label{fig:fig10}
\end{figure}

\hypertarget{tables}{%
\section{Tables}\label{tables}}

\hypertarget{references}{%
\section{References}\label{references}}

\begingroup
\setlength{\parindent}{-0.5in}
\setlength{\leftskip}{0.5in}

\hypertarget{refs}{}
\begin{cslreferences}
\leavevmode\hypertarget{ref-Agyapong-Badu2016}{}%
Agyapong-Badu, S., Warner, M., Samuel, D., and Stokes, M. (2016). Measurement of ageing effects on muscle tone and mechanical properties of rectus femoris and biceps brachii in healthy males and females using a novel hand-held myometric device. \emph{Archives of Gerontology and Geriatrics} 62, 59--67. doi:\href{https://doi.org/10.1016/j.archger.2015.09.011}{10.1016/j.archger.2015.09.011}.

\leavevmode\hypertarget{ref-Aird2012}{}%
Aird, L., Samuel, D., and Stokes, M. (2012). Quadriceps muscle tone, elasticity and stiffness in older males: Reliability and symmetry using the MyotonPRO. \emph{Archives of Gerontology and Geriatrics} 55, e31--e39. doi:\href{https://doi.org/https://doi.org/10.1016/j.archger.2012.03.005}{https://doi.org/10.1016/j.archger.2012.03.005}.

\leavevmode\hypertarget{ref-Andersen1988}{}%
Andersen, M. B., and Williams, J. M. (1988). A model of stress and athletic injury: Prediction and prevention. \emph{Journal of Sport and Exercise Psychology} 10, 294--306. doi:\href{https://doi.org/10.1123/jsep.10.3.294}{10.1123/jsep.10.3.294}.

\leavevmode\hypertarget{ref-Andersen1999}{}%
Andersen, M. B., and Williams, J. M. (1999). Athletic injury, psychosocial factors and perceptual changes during stress. \emph{Journal of Sports Sciences} 17, 735--741. doi:\href{https://doi.org/10.1080/026404199365597}{10.1080/026404199365597}.

\leavevmode\hypertarget{ref-Appaneal2009}{}%
Appaneal, R. N., Levine, B. R., Perna, F. M., and Roh, J. L. (2009). Measuring postinjury depression among male and female competitive athletes. \emph{Journal of Sport and Exercise Psychology} 31, 60--76. Available at: \url{http://www.ncbi.nlm.nih.gov/pubmed/19325188}.

\leavevmode\hypertarget{ref-Appaneal2014}{}%
Appaneal, R. N., and Perna, F. M. (2014). ``Biopsychosocial model of injury,'' in \emph{Encyclopedia of sport and exercise psychology}, eds. R. C. Eklund and G. Tenenbaum (Thousand Oaks, CA: Sage), 74--77.

\leavevmode\hypertarget{ref-Bellenger2016}{}%
Bellenger, C. R., Fuller, J. T., Thomson, R. L., Davison, K., Robertson, E. Y., and Buckley, J. D. (2016). Monitoring athletic training status through autonomic heart rate regulation: A systematic review and meta-analysis. \emph{Sports Medicine} 46, 1461--1486. doi:\href{https://doi.org/10.1007/s40279-016-0484-2}{10.1007/s40279-016-0484-2}.

\leavevmode\hypertarget{ref-Bittencourt2016}{}%
Bittencourt, N. F. N., Meeuwisse, W. H., Mendonça, L. D., Nettel-Aguirre, A., Ocarino, J. M., and Fonseca, S. T. (2016). Complex systems approach for sports injuries: Moving from risk factor identification to injury pattern recognition - narrative review and new concept. \emph{British Journal of Sports Medicine} 50, 1309--1314. doi:\href{https://doi.org/10.1136/bjsports-2015-095850}{10.1136/bjsports-2015-095850}.

\leavevmode\hypertarget{ref-Brewer2012}{}%
Brewer, B. W. (2012). ``Psychology of sport injury rehabilitation,'' in \emph{Handbook of sport psychology}, eds. G. Tenenbaum and R. C. Eklund (Hoboken, NJ, USA: Wiley), 404--424. doi:\href{https://doi.org/10.1002/9781118270011.ch18}{10.1002/9781118270011.ch18}.

\leavevmode\hypertarget{ref-Butler2003}{}%
Butler, R. J., Crowell, H. P., and Davis, I. M. C. (2003). Lower extremity stiffness: Implications for performance and injury. \emph{Clinical Biomechanics} 18, 511--517. doi:\href{https://doi.org/10.1016/S0268-0033(03)00071-8}{10.1016/S0268-0033(03)00071-8}.

\leavevmode\hypertarget{ref-Burkner2017a}{}%
Bürkner, P. C. (2017). brms: An R package for Bayesian multilevel models using Stan. \emph{Journal of Statistical Software} 80. doi:\href{https://doi.org/10.18637/jss.v080.i01}{10.18637/jss.v080.i01}.

\leavevmode\hypertarget{ref-Caminal2018}{}%
Caminal, P., Sola, F., Gomis, P., Guasch, E., Perera, A., Soriano, N., and Mont, L. (2018). Validity of the Polar V800 monitor for measuring heart rate variability in mountain running route conditions. \emph{European Journal of Applied Physiology} 118, 669--677. doi:\href{https://doi.org/10.1007/s00421-018-3808-0}{10.1007/s00421-018-3808-0}.

\leavevmode\hypertarget{ref-Chalmers2014}{}%
Chalmers, J. A., Quintana, D. S., Abbott, M. J.-A., and Kemp, A. H. (2014). Anxiety disorders are associated with reduced heart rate variability: A meta-analysis. \emph{Frontiers in psychiatry} 5, 80. doi:\href{https://doi.org/10.3389/fpsyt.2014.00080}{10.3389/fpsyt.2014.00080}.

\leavevmode\hypertarget{ref-Chen2012}{}%
Chen, S. H., and Pollino, C. A. (2012). Good practice in Bayesian network modelling. \emph{Environmental Modelling and Software} 37, 134--145. doi:\href{https://doi.org/10.1016/j.envsoft.2012.03.012}{10.1016/j.envsoft.2012.03.012}.

\leavevmode\hypertarget{ref-Chuang2013}{}%
Chuang, L. L., Lin, K. C., Wu, C. Y., Chang, C. W., Chen, H. C., Yin, H. P., and Wang, L. (2013). Relative and absolute reliabilities of the myotonometric measurements of hemiparetic arms in patients with stroke. \emph{Archives of Physical Medicine and Rehabilitation} 94, 459--466. doi:\href{https://doi.org/10.1016/j.apmr.2012.08.212}{10.1016/j.apmr.2012.08.212}.

\leavevmode\hypertarget{ref-Chuang2012}{}%
Chuang, L. L., Wu, C. Y., and Lin, K. C. (2012). Reliability, validity, and responsiveness of myotonometric measurement of muscle tone, elasticity, and stiffness in patients with stroke. \emph{Archives of Physical Medicine and Rehabilitation} 93, 532--540. doi:\href{https://doi.org/10.1016/j.apmr.2011.09.014}{10.1016/j.apmr.2011.09.014}.

\leavevmode\hypertarget{ref-VanCingel2017}{}%
Cingel, R. E. H. van, Hoogeboom, T. J., Melick, N. van, Meddeler, B. M., and Nijhuis-van der Sanden, M. W. G. (2017). How to determine leg dominance: The agreement between self-reported and observed performance in healthy adults. \emph{Plos One} 12, 1--9. doi:\href{https://doi.org/10.1371/journal.pone.0189876}{10.1371/journal.pone.0189876}.

\leavevmode\hypertarget{ref-Corr2013}{}%
Corr, P. J. (2013). Approach and avoidance behaviour: Multiple systems and their interactions. \emph{Emotion Review} 5, 285--290. doi:\href{https://doi.org/10.1177/1754073913477507}{10.1177/1754073913477507}.

\leavevmode\hypertarget{ref-Corr2016c}{}%
Corr, P. J., and Cooper, A. J. (2016). The reinforcement sensitivity theory of personality questionnaire (RST-PQ): Development and validation. \emph{Psychological Assessment} 28, 1427--1440. doi:\href{https://doi.org/10.1037/pas0000273}{10.1037/pas0000273}.

\leavevmode\hypertarget{ref-Corr2016b}{}%
Corr, P. J., McNaughton, N., Wilson, M. R., Hutchison, A., Burch, G., and Poropat, A. (2016). Neuroscience of motivation and organizational behavior: Putting the reinforcement sensitivity theory (RST) to work. \emph{Advances in Motivation and Achievement} 19, 65--92. doi:\href{https://doi.org/10.1108/S0749-742320160000019010}{10.1108/S0749-742320160000019010}.

\leavevmode\hypertarget{ref-Denver2007}{}%
Denver, J. W., Reed, S. F., and Porges, S. W. (2007). Methodological issues in the quantification of respiratory sinus arrhythmia. \emph{Biological Psychology} 74, 286--294. doi:\href{https://doi.org/10.1016/j.biopsycho.2005.09.005}{10.1016/j.biopsycho.2005.09.005}.

\leavevmode\hypertarget{ref-Djaoui2017}{}%
Djaoui, L., Haddad, M., Chamari, K., and Dellal, A. (2017). Monitoring training load and fatigue in soccer players with physiological markers. \emph{Physiology and Behavior} 181, 86--94. doi:\href{https://doi.org/10.1016/j.physbeh.2017.09.004}{10.1016/j.physbeh.2017.09.004}.

\leavevmode\hypertarget{ref-Efron1993}{}%
Efron, B., and Tibshirani, R. J. (1994). \emph{An introduction to the bootstrap}. New York: Chapman \& Hall Available at: \url{https://www.crcpress.com/An-Introduction-to-the-Bootstrap/Efron-Tibshirani/p/book/9780412042317}.

\leavevmode\hypertarget{ref-Enders2003}{}%
Enders, C. K. (2003). Using the expectation maximization algorithm to estimate coefficient alpha for scales with item-level missing data. \emph{Psychological Methods} 8, 322--337. doi:\href{https://doi.org/10.1037/1082-989X.8.3.322}{10.1037/1082-989X.8.3.322}.

\leavevmode\hypertarget{ref-Fawkner1999}{}%
Fawkner, H. J., McMurrary, N. E., and Summers, J. J. (1999). Athletic injury and minor life events: A prospective study. \emph{Journal of Science and Medicine in Sport} 2, 117--124. doi:\href{https://doi.org/10.1016/S1440-2440(99)80191-1}{10.1016/S1440-2440(99)80191-1}.

\leavevmode\hypertarget{ref-Fornell1981}{}%
Fornell, C., and Larcker, D. F. (1981). Evaluating structural equation models with unobservable variables and measurement error. \emph{Journal of Marketing Research} 18, 39--50. doi:\href{https://doi.org/10.2307/3151312}{10.2307/3151312}.

\leavevmode\hypertarget{ref-Fuller2006}{}%
Fuller, C. W., Ekstrand, J., Junge, A., Andersen, T. E., Bahr, R., Dvorak, J., Hägglund, M., McCrory, P., and Meeuwisse, W. H. (2006). Consensus statement on injury definitions and data collection procedures in studies of football (soccer) injuries. \emph{British journal of sports medicine} 40, 193--201. doi:\href{https://doi.org/10.1136/bjsm.2005.025270}{10.1136/bjsm.2005.025270}.

\leavevmode\hypertarget{ref-Fuller2007b}{}%
Fuller, C. W., Molloy, M. G., Bagate, C., Bahr, R., Brooks, J. H. M., Donson, H., Kemp, S. P. T., McCrory, P., McIntosh, A. S., Meeuwisse, W. H., et al. (2007). Consensus statement on injury definitions and data collection procedures for studies of injuries in rugby union. \emph{British journal of sports medicine} 41, 328--31. doi:\href{https://doi.org/10.1136/bjsm.2006.033282}{10.1136/bjsm.2006.033282}.

\leavevmode\hypertarget{ref-Fuster-Parra2017}{}%
Fuster-Parra, P., Vidal-Conti, J., Borràs, P. A., and Palou, P. (2017). Bayesian networks to identify statistical dependencies. A case study of Spanish university students' habits. \emph{Informatics for Health and Social Care} 42, 166--179. doi:\href{https://doi.org/10.1080/17538157.2016.1178117}{10.1080/17538157.2016.1178117}.

\leavevmode\hypertarget{ref-Galambos2005}{}%
Galambos, S. A., Terry, P. C., Moyle, G. M., and Locke, S. A. (2005). Psychological predictors of injury among elite athletes. \emph{British Journal of Sports Medicine} 39, 351--354. doi:\href{https://doi.org/10.1136/bjsm.2005.018440}{10.1136/bjsm.2005.018440}.

\leavevmode\hypertarget{ref-Giles2016}{}%
Giles, D., Draper, N., and Neil, W. (2016). Validity of the Polar V800 heart rate monitor to measure RR intervals at rest. \emph{European Journal of Applied Physiology} 116, 563--571. doi:\href{https://doi.org/10.1007/s00421-015-3303-9}{10.1007/s00421-015-3303-9}.

\leavevmode\hypertarget{ref-Gisselman2016}{}%
Gisselman, A. S., Baxter, G. D., Wright, A., Hegedus, E., and Tumilty, S. (2016). Musculoskeletal overuse injuries and heart rate variability: Is there a link? \emph{Medical Hypotheses} 87, 1--7. doi:\href{https://doi.org/10.1016/j.mehy.2015.12.003}{10.1016/j.mehy.2015.12.003}.

\leavevmode\hypertarget{ref-Heckerman1995}{}%
Heckerman, D., Geiger, D., and Chickering, D. M. (1995). Learning Bayesian networks: The combination of knowledge and statistical data. \emph{Machine Learning} 20, 197--243. doi:\href{https://doi.org/10.1023/A:1022623210503}{10.1023/A:1022623210503}.

\leavevmode\hypertarget{ref-Hrysomallis2011}{}%
Hrysomallis, C. (2011). Balance ability and athletic performance. \emph{Sports Medicine} 41, 221--232. doi:\href{https://doi.org/10.2165/11538560-000000000-00000}{10.2165/11538560-000000000-00000}.

\leavevmode\hypertarget{ref-Hughes2014}{}%
Hughes, G. (2014). A review of recent perspectives on biomechanical risk factors associated with anterior cruciate ligament injury. \emph{Research in Sports Medicine} 22, 193--212. doi:\href{https://doi.org/10.1080/15438627.2014.881821}{10.1080/15438627.2014.881821}.

\leavevmode\hypertarget{ref-Hunt2009}{}%
Hunt, T. N., Ferrara, M. S., Bornstein, R. A., and Baumgartner, T. A. (2009). The reliability of the modified balance error scoring system. \emph{Clinical Journal of Sport Medicine} 19, 471--475. doi:\href{https://doi.org/10.1097/JSM.0b013e3181c12c7b}{10.1097/JSM.0b013e3181c12c7b}.

\leavevmode\hypertarget{ref-Ivarsson2010}{}%
Ivarsson, A., and Johnson, U. (2010). Psychological factors as predictors of injuries among senior soccer players. A prospective study. \emph{Journal of Sports Science and Medicine} 9, 347--352. Available at: \url{https://www.ncbi.nlm.nih.gov/pmc/articles/PMC3761721/}.

\leavevmode\hypertarget{ref-Ivarsson2017}{}%
Ivarsson, A., Johnson, U., Andersen, M. B., Tranaeus, U., Stenling, A., and Lindwall, M. (2017). Psychosocial factors and sport injuries: Meta-analyses for prediction and prevention. \emph{Sports Medicine} 47, 353--365. doi:\href{https://doi.org/10.1007/s40279-016-0578-x}{10.1007/s40279-016-0578-x}.

\leavevmode\hypertarget{ref-Ivarsson2014}{}%
Ivarsson, A., Johnson, U., Lindwall, M., Gustafsson, H., and Altemyr, M. (2014). Psychosocial stress as a predictor of injury in elite junior soccer: A latent growth curve analysis. \emph{Journal of Science and Medicine in Sport} 17, 366--370. doi:\href{https://doi.org/10.1016/j.jsams.2013.10.242}{10.1016/j.jsams.2013.10.242}.

\leavevmode\hypertarget{ref-Johnson2014}{}%
Johnson, U., Tranaeus, U., and Ivarsson, A. (2014). Current status and future challenges in psychological research of sport injury prediction and prevention: A methodological perspective. \emph{Revista de Psicologia del Deporte} 23, 401--409. Available at: \url{https://www.rpd-online.com/article/view/v23-n2-johnson-tranaeus-ivarsson}.

\leavevmode\hypertarget{ref-Kalkhoven2018}{}%
Kalkhoven, J. T., and Watsford, M. L. (2017). The relationship between mechanical stiffness and athletic performance markers in sub-elite footballers. \emph{Journal of Sports Sciences} 36, 1022--1029. doi:\href{https://doi.org/10.1080/02640414.2017.1349921}{10.1080/02640414.2017.1349921}.

\leavevmode\hypertarget{ref-Koller2009}{}%
Koller, D., and Friedman, N. (2009). \emph{Probabilistic graphical models: Principles and techniques - Adaptive computation and machine learning}. Cambridge, MA: The MIT Press doi:\href{https://doi.org/10.1017/CBO9781107415324.004}{10.1017/CBO9781107415324.004}.

\leavevmode\hypertarget{ref-Koo2016}{}%
Koo, T. K., and Li, M. Y. (2016). A guideline of selecting and reporting intraclass correlation coefficients for reliability research. \emph{Journal of Chiropractic Medicine} 15, 155--63. doi:\href{https://doi.org/10.1016/j.jcm.2016.02.012}{10.1016/j.jcm.2016.02.012}.

\leavevmode\hypertarget{ref-Kuhn2008}{}%
Kuhn, M. C. from J. W., Weston, S., Williams, A., Keefer, C., and Engelhardt, A. (2008). caret: Classification and Regression Training. \emph{Journal of Statistical Software} 28. doi:\href{https://doi.org/10.1053/j.sodo.2009.03.002}{10.1053/j.sodo.2009.03.002}.

\leavevmode\hypertarget{ref-Kumar2001}{}%
Kumar, S. (2001). Theories of musculoskeletal injury causation. \emph{Ergonomics} 44, 17--47. doi:\href{https://doi.org/10.1080/00140130120716}{10.1080/00140130120716}.

\leavevmode\hypertarget{ref-Laborde2017}{}%
Laborde, S., Mosley, E., and Thayer, J. F. (2017). Heart rate variability and cardiac vagal tone in psychophysiological research - Recommendations for experiment planning, data analysis, and data reporting. \emph{Frontiers in Psychology} 8, 1--18. doi:\href{https://doi.org/10.3389/fpsyg.2017.00213}{10.3389/fpsyg.2017.00213}.

\leavevmode\hypertarget{ref-Lang2014}{}%
Lang, K. M., Jorgensen, T. D., Moore, E. W. G., and Little, T. D. (2013). On the joys of missing data. \emph{Journal of Pediatric Psychology} 39, 151--162. doi:\href{https://doi.org/10.1093/jpepsy/jst048}{10.1093/jpepsy/jst048}.

\leavevmode\hypertarget{ref-Lavallee1996}{}%
Lavallée, L., and Flint, F. (1996). The relationship of stress, competitive anxiety, mood state, and social support to athletic injury. \emph{Journal of Athletic Training} 31, 296--299. Available at: \url{http://www.ncbi.nlm.nih.gov/pubmed/16558413}.

\leavevmode\hypertarget{ref-Leddy1994}{}%
Leddy, M. H., Lambert, M. J., and Ogles, B. M. (1994). Psychological consequences of athletic injury among high-level competitors. \emph{Research Quarterly for Exercise and Sport} 65, 347--354. doi:\href{https://doi.org/10.1080/02701367.1994.10607639}{10.1080/02701367.1994.10607639}.

\leavevmode\hypertarget{ref-Lee2017}{}%
Lee, E. C., Fragala, M. S., Kavouras, S. A., Queen, R. M., Pryor, J. L., and Casa, D. J. (2017). Biomarkers in sports and exercise: Tracking health, performance, and recovery in athletes. \emph{Journal of Strength and Conditioning Research} 31, 2920--2937. doi:\href{https://doi.org/10.1519/JSC.0000000000002122}{10.1519/JSC.0000000000002122}.

\leavevmode\hypertarget{ref-Maddison2005}{}%
Maddison, R., and Prapavessis, H. (2005). A psychological approach to the prediction and prevention of athletic injury. \emph{Journal of Sport and Exercise Psychology} 27, 289--310. doi:\href{https://doi.org/10.1123/jsep.27.3.289}{10.1123/jsep.27.3.289}.

\leavevmode\hypertarget{ref-Malik1996}{}%
Malik, M., Camm, A. J., Bigger, J. T., Breithardt, G., Cerutti, S., Cohen, R. J., Coumel, P., Fallen, E. L., Kennedy, H. L., Kleiger, R. E., et al. (1996). Heart rate variability. Standards of measurement, physiological interpretation, and clinical use. \emph{European Heart Journal} 17, 354--381. doi:\href{https://doi.org/10.1093/oxfordjournals.eurheartj.a014868}{10.1093/oxfordjournals.eurheartj.a014868}.

\leavevmode\hypertarget{ref-McCrory2013}{}%
McCrory, P., Meeuwisse, W. H., Aubry, M., Cantu, R. C., Dvořák, J., Echemendia, R. J., Engebretsen, L., Johnston, K., Kutcher, J. S., Raftery, M., et al. (2013). Consensus statement on concussion in sport: The 4th international conference on concussion in sport, Zurich, November 2012. \emph{Journal of Athletic Training} 48, 554--575. doi:\href{https://doi.org/10.4085/1062-6050-48.4.05}{10.4085/1062-6050-48.4.05}.

\leavevmode\hypertarget{ref-Meeuwisse2007}{}%
Meeuwisse, W. H., Tyreman, H., Hagel, B., and Emery, C. (2007). A dynamic model of etiology in sport injury: The recursive nature of risk and causation. \emph{Clinical Journal of Sport Medicine} 17, 215--219. doi:\href{https://doi.org/10.1097/JSM.0b013e3180592a48}{10.1097/JSM.0b013e3180592a48}.

\leavevmode\hypertarget{ref-Morgan2018}{}%
Morgan, G. E., Martin, R., Williams, L., Pearce, O., and Morris, K. (2018). Objective assessment of stiffness in Achilles tendinopathy: a novel approach using the MyotonPRO. \emph{BMJ Open Sport \& Exercise Medicine} 4, e000446. doi:\href{https://doi.org/10.1136/bmjsem-2018-000446}{10.1136/bmjsem-2018-000446}.

\leavevmode\hypertarget{ref-Murphy2003}{}%
Murphy, D. F., Connolly, D. A. J., and Beynnon, B. D. (2003). Risk factors for lower extremity injury: A review of the literature. \emph{British Journal of Sports Medicine} 37, 13--29. doi:\href{https://doi.org/10.1136/bjsm.37.1.13}{10.1136/bjsm.37.1.13}.

\leavevmode\hypertarget{ref-Murphy2002}{}%
Murphy, K. (2002). Dynamic Bayesian networks: Representation, inference and learning.

\leavevmode\hypertarget{ref-Nair2014}{}%
Nair, K., Dougherty, J., Schaefer, E., Kelly, J., and Masi, A. (2014). Repeatability, reproducibility, and calibration of the MyotonPRO on tissue mimicking phantoms. in \emph{ASME summer bioengineering conference}, 1--2. doi:\href{https://doi.org/10.1115/SBC2013-14622}{10.1115/SBC2013-14622}.

\leavevmode\hypertarget{ref-Neely1998}{}%
Neely, F. G. (1998). Biomechanical risk factors for exercise-related lower limb injuries. \emph{Sports Medicine} 26, 395--413. doi:\href{https://doi.org/10.2165/00007256-199826060-00003}{10.2165/00007256-199826060-00003}.

\leavevmode\hypertarget{ref-Olmedilla2018}{}%
Olmedilla, A., Rubio, V. J., Fuster-Parra, P., Pujals, C., and García-Mas, A. (2018). A Bayesian approach to sport injuries likelihood: Does player's self-efficacy and environmental factors plays the main role? \emph{Frontiers in Psychology} 9, 1--10. doi:\href{https://doi.org/10.3389/fpsyg.2018.01174}{10.3389/fpsyg.2018.01174}.

\leavevmode\hypertarget{ref-Paillard2006}{}%
Paillard, T., Noé, F., Rivière, T., Marion, V., Montoya, R., and Dupui, P. (2006). Postural performance and strategy in the unipedal stance of soccer players at different levels of competition. \emph{Journal of Athletic Training} 41, 172--176. Available at: \url{http://www.ncbi.nlm.nih.gov/pubmed/16791302}.

\leavevmode\hypertarget{ref-Passer1983a}{}%
Passer, M. W., and Seese, M. D. (1983). Life stress and athletic injury: Examination of positive versus negative events and three moderator variables. \emph{Journal of Human Stress} 9, 11--16. doi:\href{https://doi.org/10.1080/0097840X.1983.9935025}{10.1080/0097840X.1983.9935025}.

\leavevmode\hypertarget{ref-Pearl1988}{}%
Pearl, J. (1988). \emph{Probabilistic reasoning in intelligent systems: Networks of plausible inference}. San Francisco, CA: Morgan Kaufmann Publishers.

\leavevmode\hypertarget{ref-Perna2003}{}%
Perna, F. M., Antoni, M. H., Baum, A., Gordon, P., and Schneiderman, N. (2003). Cognitive behavioral stress management effects on injury and illness among competitive athletes: A randomized clinical trial. \emph{Annals of Behavioral Medicine} 25, 66--73. Available at: \url{https://www.scopus.com/inward/record.uri?eid=2-s2.0-0037262738\%7B/\&\%7DpartnerID=40\%7B/\&\%7Dmd5=37332ebe6a962abf5178e51d6ebaedb7}.

\leavevmode\hypertarget{ref-Perna1995}{}%
Perna, F. M., and McDowell, S. L. (1995). Role of psychological stress in cortisol recovery from exhaustive exercise among elite athletes. \emph{International Journal of Behavioral Medicine} 2, 13--26. doi:\href{https://doi.org/10.1207/s15327558ijbm0201_2}{10.1207/s15327558ijbm0201\_2}.

\leavevmode\hypertarget{ref-Perna1997}{}%
Perna, F., Schneiderman, N., and LaPerriere, A. (1997). Psychological stress, exercise and immunity. \emph{International Journal of Sports Medicine} 18, 78--83. doi:\href{https://doi.org/10.1055/s-2007-972703}{10.1055/s-2007-972703}.

\leavevmode\hypertarget{ref-Petrie1993}{}%
Petrie, T. A. (1993). Coping skills, competitive trait anxiety, and playing status: Moderating effects on the life stress-injury relationship. \emph{Journal of Sport and Exercise Psychology} 15, 261--274. doi:\href{https://doi.org/10.1123/jsep.15.3.261}{10.1123/jsep.15.3.261}.

\leavevmode\hypertarget{ref-Petrie1992}{}%
Petrie, T. A. (1992). Psychosocial antecedents of athletic injury: The effects of life stress and social support on female collegiate gymnasts. \emph{Journal of Behavioral Medicine} 18, 127--138. doi:\href{https://doi.org/10.1080/08964289.1992.9936963}{10.1080/08964289.1992.9936963}.

\leavevmode\hypertarget{ref-Petrie2014}{}%
Petrie, T. A., Deiters, J., and Harmison, R. J. (2014). Mental toughness, social support, and athletic identity: Moderators of the life stress-injury relationship in collegiate football players. \emph{Sport, Exercise, and Performance Psychology} 3, 13--27. doi:\href{https://doi.org/10.1037/a0032698}{10.1037/a0032698}.

\leavevmode\hypertarget{ref-PickeringRodriguez2017}{}%
Pickering-Rodriguez, E. C., Watsford, M. L., Bower, R. G., and Murphy, A. J. (2017). The relationship between lower body stiffness and injury incidence in female netballers. \emph{Sports Biomechanics} 16, 361--373. doi:\href{https://doi.org/10.1080/14763141.2017.1319970}{10.1080/14763141.2017.1319970}.

\leavevmode\hypertarget{ref-Plews2017}{}%
Plews, D. J., Scott, B., Altini, M., Wood, M., Kilding, A. E., and Laursen, P. B. (2017). Comparison of heart-rate-variability recording with smartphone photoplethysmography, polar H7 chest strap, and electrocardiography. \emph{International Journal of Sports Physiology and Performance} 12, 1324--1328. doi:\href{https://doi.org/10.1123/ijspp.2016-0668}{10.1123/ijspp.2016-0668}.

\leavevmode\hypertarget{ref-Pruyn2015}{}%
Pruyn, E. C., Watsford, M. L., and Murphy, A. J. (2015). Differences in lower-body stiffness between levels of netball competition. \emph{Journal of Strength and Conditioning Research} 29, 1197--1202. doi:\href{https://doi.org/10.1519/JSC.0000000000000418}{10.1519/JSC.0000000000000418}.

\leavevmode\hypertarget{ref-Pruyn2016}{}%
Pruyn, E. C., Watsford, M. L., and Murphy, A. J. (2016). Validity and reliability of three methods of stiffness assessment. \emph{Journal of Sport and Health Science} 5, 476--483. doi:\href{https://doi.org/10.1016/j.jshs.2015.12.001}{10.1016/j.jshs.2015.12.001}.

\leavevmode\hypertarget{ref-Pulopulos2018}{}%
Pulopulos, M. M., Vanderhasselt, M. A., and De Raedt, R. (2018). Association between changes in heart rate variability during the anticipation of a stressful situation and the stress-induced cortisol response. \emph{Psychoneuroendocrinology} 94, 63--71. doi:\href{https://doi.org/10.1016/j.psyneuen.2018.05.004}{10.1016/j.psyneuen.2018.05.004}.

\leavevmode\hypertarget{ref-Qian2015}{}%
Qian, S. S., and Miltner, R. J. (2015). A continuous variable Bayesian networks model for water quality modeling: A case study of setting nitrogen criterion for small rivers and streams in Ohio, USA. \emph{Environmental Modelling and Software} 69, 14--22. doi:\href{https://doi.org/10.1016/j.envsoft.2015.03.001}{10.1016/j.envsoft.2015.03.001}.

\leavevmode\hypertarget{ref-RCoreTeam2019}{}%
R Core Team (2019). \emph{R: A language and environment for statistical computing}. Vienna, Austria: R Foundation for Statistical Computing Available at: \url{https://www.r-project.org/}.

\leavevmode\hypertarget{ref-Riemann1999d}{}%
Riemann, B. L., Guskiewicz, K. M., and Shields, E. W. (1999). Relationship between clinical and forceplate measures of postural stability. \emph{Journal of Sport Rehabilitation} 8, 71--82. doi:\href{https://doi.org/10.1123/jsr.8.2.71}{10.1123/jsr.8.2.71}.

\leavevmode\hypertarget{ref-Rodriguez-Linares2017}{}%
Rodriguez-Linares, L., Vila, X., Lado, M. J., Mendez, A., Otero, A., and Garcia, C. A. (2019). \emph{RHRV: Heart rate variability analysis of ecg data}. Available at: \url{https://CRAN.R-project.org/package=RHRV}.

\leavevmode\hypertarget{ref-Rogers2005}{}%
Rogers, T., and Landers, D. M. (2005). Mediating effects of peripheral vision in the life event stress/athletic injury relationship. \emph{Journal of Sport and Exercise Psychology} 27, 271--288. doi:\href{https://doi.org/10.1002/9781444303650}{10.1002/9781444303650}.

\leavevmode\hypertarget{ref-Romero-Franco2014}{}%
Romero-Franco, N., Gallego-Izquierdo, T., Martínez-López, E. J., Hita-Contreras, F., Osuna-Pére, Catalina, M., and Martínez-Amat, A. (2014). Postural stability and subsequent sports injuries during indoor season of athletes. \emph{Journal of Physical Therapy Science} 26, 683--687. doi:\href{https://doi.org/10.1589/jpts.26.683}{10.1589/jpts.26.683}.

\leavevmode\hypertarget{ref-Rosa2014}{}%
Rosa, B. B., Asperti, A. M., Helito, C. P., Demange, M. K., Fernandes, T. L., and Hernandez, A. J. (2014). Epidemiology of sports injuries on collegiate athletes at a single center. \emph{Acta Ortopédica Brasileira} 22, 321--324. doi:\href{https://doi.org/10.1590/1413-78522014220601007}{10.1590/1413-78522014220601007}.

\leavevmode\hypertarget{ref-Ross2011}{}%
Ross, S. E., Linens, S. W., Wright, C. J., and Arnold, B. L. (2011). Balance assessments for predicting functional ankle instability and stable ankles. \emph{Gait and Posture} 34, 539--542. doi:\href{https://doi.org/10.1016/j.gaitpost.2011.07.011}{10.1016/j.gaitpost.2011.07.011}.

\leavevmode\hypertarget{ref-Ruiz-Perez2019}{}%
Ruiz-Pérez, I., Ayala, F., Miguel Puerta, J., Elvira, J. L. L., De Ste Croix, M., Hernández-Sánchez, S., and Vera-Garcia, F. J. (2019). A Bayesian Network approach to study the relationships between several neuromuscular performance measures and dynamic postural control in futsal players. \emph{PLoS ONE} 14, 1--18. doi:\href{https://doi.org/10.1371/journal.pone.0220065}{10.1371/journal.pone.0220065}.

\leavevmode\hypertarget{ref-Norvig2009}{}%
Russell, S. J., and Norvig, P. (2009). \emph{Artificial Intelligence: A Modern Approach}. 3rd ed. Prentice Hall Available at: \url{http://aima.cs.berkeley.edu/}.

\leavevmode\hypertarget{ref-Scutari2014}{}%
Scutari, M., and Denis, J.-B. (2014). \emph{Bayesian networks: with examples in R}. 1st ed. Chapman \& Hall/CRC Available at: \url{https://www.crcpress.com/Bayesian-Networks-With-Examples-in-R/Scutari-Denis/p/book/9781482225587}.

\leavevmode\hypertarget{ref-Scutari2013}{}%
Scutari, M., and Nagarajan, R. (2013). Identifying significant edges in graphical models of molecular networks. \emph{Artificial Intelligence in Medicine} 57, 207--217. doi:\href{https://doi.org/10.1016/j.artmed.2012.12.006}{10.1016/j.artmed.2012.12.006}.

\leavevmode\hypertarget{ref-Sheu2016}{}%
Sheu, Y., Chen, L. H., and Hedegaard, H. (2016). Sports- and recreation-related injury episodes in the United States, 2011-2014. \emph{National Health Statistics Reports}, 1--12. Available at: \url{https://www.ncbi.nlm.nih.gov/pubmed/27906643}.

\leavevmode\hypertarget{ref-Slimani2018}{}%
Slimani, M., Bragazzi, N. L., Znazen, H., Paravlic, A., Azaiez, F., and Tod, D. (2018). Psychosocial predictors and psychological prevention of soccer injuries: A systematic review and meta-analysis of the literature. \emph{Physical Therapy in Sport} 32, 293--300. doi:\href{https://doi.org/10.1016/j.ptsp.2018.05.006}{10.1016/j.ptsp.2018.05.006}.

\leavevmode\hypertarget{ref-Smith1990}{}%
Smith, R. E., Smoll, F. L., and Ptacek, J. T. (1990). Conjunctive moderator variables in vulnerability and resiliency research: Life stress, social support and coping skills, and adolescent sport injuries. \emph{Journal of Personality and Social Psychology} 58, 360--370. doi:\href{https://doi.org/10.1037/0022-3514.58.2.360}{10.1037/0022-3514.58.2.360}.

\leavevmode\hypertarget{ref-Swanik2007}{}%
Swanik, C. B., Covassin, T., Stearne, D. J., and Schatz, P. (2007). The relationship between neurocognitive function and noncontact anterior cruciate ligament injuries. \emph{American Journal of Sports Medicine} 35, 943--948. doi:\href{https://doi.org/10.1177/0363546507299532}{10.1177/0363546507299532}.

\leavevmode\hypertarget{ref-Timpka2014}{}%
Timpka, T., Alonso, J. M., Jacobsson, J., Junge, A., Branco, P., Clarsen, B., Kowalski, J., Mountjoy, M., Nilsson, S., Pluim, B., et al. (2014). Injury and illness definitions and data collection procedures for use in epidemiological studies in Athletics (track and field): Consensus statement. \emph{British Journal of Sports Medicine} 48, 483--490. doi:\href{https://doi.org/10.1136/bjsports-2013-093241}{10.1136/bjsports-2013-093241}.

\leavevmode\hypertarget{ref-Trojian2006}{}%
Trojian, T. H., and McKeag, D. B. (2006). Single leg balance test to identify risk of ankle sprains. \emph{British Journal of Sports Medicine} 40, 610--613. doi:\href{https://doi.org/10.1136/bjsm.2005.024356}{10.1136/bjsm.2005.024356}.

\leavevmode\hypertarget{ref-Valovich2003}{}%
Valovich, T. C., Perrin, D. H., and Gansneder, B. M. (2003). Repeat administration elicits a practice effect with the Balance Error Scoring System but not with the Standardized Assessment of Concussion in high school athletes. \emph{Journal of Athletic Training} 38, 51--56. Available at: \url{https://www.ncbi.nlm.nih.gov/pubmed/12937472}.

\leavevmode\hypertarget{ref-Wiese-Bjornstal2009}{}%
Wiese-Bjornstal, D. M. (2009). Sport injury and college athlete health across the lifespan. \emph{Journal of Intercollegiate Sport} 2, 64--80. doi:\href{https://doi.org/10.1123/jis.2.1.64}{10.1123/jis.2.1.64}.

\leavevmode\hypertarget{ref-Wilkerson2012a}{}%
Wilkerson, G. B. (2012). Neurocognitive reaction time predicts lower extremity sprains and strains. \emph{International Journal of Athletic Therapy and Training} 17, 4--9. doi:\href{https://doi.org/10.1123/ijatt.17.6.4}{10.1123/ijatt.17.6.4}.

\leavevmode\hypertarget{ref-Williams2007}{}%
Williams, J. M., and Andersen, M. B. (2007). ``Psychosocial antecedents of sport injury and interventions for risk reduction,'' in \emph{Handbook of sport psychology}, eds. G. Tenenbaum and R. C. Eklund (Hoboken, NJ, USA: Wiley), 379--403. doi:\href{https://doi.org/10.1002/9781118270011.ch17}{10.1002/9781118270011.ch17}.

\leavevmode\hypertarget{ref-Williams1998}{}%
Williams, J. M., and Andersen, M. B. (1998). Psychosocial antecedents of sport injury: review and critique of the stress and injury model. \emph{Journal of Applied Sport Psychology} 10, 5--25. doi:\href{https://doi.org/10.1080/10413209808406375}{10.1080/10413209808406375}.

\leavevmode\hypertarget{ref-Williams2017}{}%
Williams, S., Booton, T., Watson, M., Rowland, D., and Altini, M. (2017). Heart rate variability is a moderating factor in the workload-injury relationship of competitive crossfit™ athletes. \emph{Journal of Sports Science and Medicine} 16, 443--449.

\leavevmode\hypertarget{ref-Xiao-xuan2007}{}%
Xiao-xuan, H., Hui, W., and Shuo, W. (2008). Using expert's knowledge to build Bayesian networks. in \emph{International conference on computational intelligence and security workshops} (IEEE), 220--223. doi:\href{https://doi.org/10.1109/cisw.2007.4425484}{10.1109/cisw.2007.4425484}.

\leavevmode\hypertarget{ref-Xu2015}{}%
Xu, J. G., Zhao, Y., Chen, J., and Han, C. (2015). A structure learning algorithm for Bayesian network using prior knowledge. \emph{Journal of Computer Science and Technology} 30, 713--724. doi:\href{https://doi.org/10.1007/s11390-015-1556-8}{10.1007/s11390-015-1556-8}.
\end{cslreferences}

\endgroup

\end{document}
