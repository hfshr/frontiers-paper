%%%%%%%%%%%%%%%%%%%%%%%%%%%%%%%%%%%%%%%%%%%%%%%%%%%%%%%%%%%%%%%%%%%%%%%%%%%%%%%%%%%%%%%%%%%%%%%%%%%%%%%%%%%%%%%%%%%%%%%%%%%%%%%%%%%%%%%%%%%%%%%%%%%%%%%%%%%
% This is just an example/guide for you to refer to when submitting manuscripts to Frontiers, it is not mandatory to use Frontiers .cls files nor frontiers.tex  %
% This will only generate the Manuscript, the final article will be typeset by Frontiers after acceptance.
%                                              %
%                                                                                                                                                         %
% When submitting your files, remember to upload this *tex file, the pdf generated with it, the *bib file (if bibliography is not within the *tex) and all the figures.
%%%%%%%%%%%%%%%%%%%%%%%%%%%%%%%%%%%%%%%%%%%%%%%%%%%%%%%%%%%%%%%%%%%%%%%%%%%%%%%%%%%%%%%%%%%%%%%%%%%%%%%%%%%%%%%%%%%%%%%%%%%%%%%%%%%%%%%%%%%%%%%%%%%%%%%%%%%

%%% Version 3.4 Generated 2018/06/15 %%%
%%% You will need to have the following packages installed: datetime, fmtcount, etoolbox, fcprefix, which are normally inlcuded in WinEdt. %%%
%%% In http://www.ctan.org/ you can find the packages and how to install them, if necessary. %%%

\documentclass[utf8]{frontiersHLTH}

%\setcitestyle{square} % for Physics and Applied Mathematics and Statistics articles
\usepackage{url,hyperref,lineno,microtype,subcaption}
\usepackage[onehalfspacing]{setspace}

\linenumbers


% BELOW TAKEN FROM rticles plos template
%
% amsmath package, useful for mathematical formulas
\usepackage{amsmath}
% amssymb package, useful for mathematical symbols
\usepackage{amssymb}

% hyperref package, useful for hyperlinks
\usepackage{hyperref}

% graphicx package, useful for including eps and pdf graphics
% include graphics with the command \includegraphics
\usepackage{graphicx}

% Sweave(-like)
\usepackage{fancyvrb}
\DefineVerbatimEnvironment{Sinput}{Verbatim}{fontshape=sl}
\DefineVerbatimEnvironment{Soutput}{Verbatim}{}
\DefineVerbatimEnvironment{Scode}{Verbatim}{fontshape=sl}
\newenvironment{Schunk}{}{}
\DefineVerbatimEnvironment{Code}{Verbatim}{}
\DefineVerbatimEnvironment{CodeInput}{Verbatim}{fontshape=sl}
\DefineVerbatimEnvironment{CodeOutput}{Verbatim}{}
\newenvironment{CodeChunk}{}{}

% cite package, to clean up citations in the main text. Do not remove.
\usepackage{cite}

\usepackage{color}

\providecommand{\tightlist}{%
  \setlength{\itemsep}{0pt}\setlength{\parskip}{0pt}}

% Below is from frontiers
%
\bibliographystyle{frontiersinSCNS}
% Use doublespacing - comment out for single spacing
%\usepackage{setspace}
%\doublespacing


% Leave a blank line between paragraphs instead of using \\


\def\keyFont{\fontsize{8}{11}\helveticabold }


%% ** EDIT HERE **
%% PLEASE INCLUDE ALL MACROS BELOW

%% END MACROS SECTION


\usepackage{booktabs}
\usepackage{longtable}
\usepackage{array}
\usepackage{multirow}
\usepackage{wrapfig}
\usepackage{float}
\usepackage{colortbl}
\usepackage{pdflscape}
\usepackage{tabu}
\usepackage{threeparttable}
\usepackage{threeparttablex}
\usepackage[normalem]{ulem}
\usepackage{makecell}
\usepackage{xcolor}


\def\Authors{
  Dr Harry Fisher\,\textsuperscript{1*},
  Dr Marianne Gittoes\,\textsuperscript{1},
  Professor Lynne Evans\,\textsuperscript{1},
  Miss Leah Bitchell\,\textsuperscript{1},
  Dr Richard Mullen\,\textsuperscript{2},
  Dr Marco Scutari\,\textsuperscript{3}}

\def\Address{

  \textsuperscript{1} School of Sport and Health Science, Cardiff Metropolitan University,  Cardiff,   United Kingdom
  
  \textsuperscript{2} Division of Sport, Health \& Exercise Sciences, Brunel University,  London,   United Kingdom
  
  \textsuperscript{3} Istituto Dalle Molle di Studi sull'Intelligenza Artificiale (IDSIA),  Manno,   Switzerland
  }

  \def\corrAuthor{Dr Harry Fisher}\def\corrAddress{Cardiff Metropolitan University\\Cyncoed Road\\, CF23 6XD United Kingdom}\def\corrEmail{\href{mailto:harryfisher21@gmail.com}{\nolinkurl{harryfisher21@gmail.com}}}
  \def\firstAuthorLast{Fisher {et~al.}}
  
  
  
  
  
  
  
  
  
  


\begin{document}
\onecolumn
\firstpage{1}

\title[Stress and sports injury]{An interdisciplinary examination of stress and injury occurrence in
athletes}
\author[\firstAuthorLast]{\Authors}
\address{} %This field will be automatically populated
\correspondance{} %This field will be automatically populated

\extraAuth{}% If there are more than 1 corresponding author, comment this line and uncomment the next one.
%\extraAuth{corresponding Author2 \\ Laboratory X2, Institute X2, Department X2, Organization X2, Street X2, City X2 , State XX2 (only USA, Canada and Australia), Zip Code2, X2 Country X2, email2@uni2.edu}


\maketitle

\begin{abstract}

This paper adopts a novel, interdisciplinary approach to explore the relationship between psychosocial factors, physiological stress-related markers and occurrence of injury in athletes using a repeated measures design across a 2-year data collection period. At three data collection time-points, athletes completed measures of major life events, the reinforcement sensitivity theory personality questionnaire, muscle stiffness, heart rate variability and postural stability, an
d reported any injuries they had sustained since the last data collection. Two Bayesian networks were used to examine the relationships between variables and model the changes between data collection points in the study. Findings revealed muscle stiffness to have the strongest relationship with injury occurrence, with high levels of stiffness increasing the probability of sustaining an injury. Negative life events did not increase the probability of injury occurrence at any single time-point; however, when examining changes between time points, increases in negative life events did increase the probability of injury. In addition, the combination of increases in negative life events and muscle stiffness resulted in the greatest probability of sustaining an injury. Findings demonstrated the importance of both an interdisciplinary approach and a repeated measures design to furthering our understanding of the relationship between stress-related  markers and injury occurrence.

\tiny
 \keyFont{ \section{Keywords:} Sports Injury, Stress, Interdisciplinary, Bayesian Network} 

\end{abstract}

\hypertarget{introduction}{%
\section*{Introduction}\label{introduction}}
\addcontentsline{toc}{section}{Introduction}

Over the last four decades sport related injuries have received
increased research attention (Ivarsson et al., 2017). This attention is
unsurprising given the high incidence (Rosa et al., 2014; Sheu et al.,
2016), and undesirable physical and psychological effects of sports
injuries (Leddy et al., 1994; Brewer, 2012). To mitigate against both
the increasing incidence and undesirable consequences of injury,
research has identified several psychological (Slimani et al., 2018),
anatomical (Murphy et al., 2003), biomechanical (Neely, 1998; Hughes,
2014) and environmental (Meeuwisse et al., 2007) factors associated with
sports injury occurrence. Indeed, several models of injury causation
have been proposed that highlight the multifactorial nature of injury
occurrence (Kumar, 2001; Meeuwisse et al., 2007; Wiese-Bjornstal, 2009),
of which one of the most widely cited was developed by Williams and
Anderson (Fig \ref{fig:fig1}; Andersen and Williams, 1988; Williams and
Andersen, 1998).

\begin{figure}[!h]
\caption{{\bf Stress and injury model (Williams and Andersen, 1998).}}
\label{fig:fig1}
\end{figure}

Williams and Andersen's (Williams and Andersen, 1998) model proposed
that when faced with a potentially stressful athletic situation, an
athlete's personality traits (e.g., hardiness, locus of control and
competitive trait anxiety), history of stressors (e.g., major life
events and previous injuries) and coping resources (e.g., general coping
behaviours) will contribute to their response, either interactively or
in isolation. Central to the model is the stress response, which
reflects the bi-directional relationship between athletes' appraisal of,
and response to, a stressful athletic situation. The model predicts that
athletes who have a history of many stressors, personality traits that
intensify the stress response and few coping resources, will exhibit
greater attentional (e.g., peripheral narrowing) and/or physiological
(e.g., increased muscle tension) responses that put these individuals at
greater risk of injury.

Within Williams and Andersen's (Williams and Andersen, 1998) model,
major life events, a component of an athlete's history of stressors,
most consistently predicts injury occurrence (Williams and Andersen,
2007); specifically, major life events with a negative, as opposed to
positive, valence (Passer and Seese, 1983; Maddison and Prapavessis,
2005). However, personality traits and coping resources have also been
found to predict injury, with for example, athletes more likely to
sustain an injury if they have poor social support and psychological
coping skills, and high trait anxiety and elevated competitive state
anxiety; compared to athletes with the opposite profile. (Smith et al.,
1990; Lavallée and Flint, 1996; Ivarsson and Johnson, 2010). However,
the amount of variance explained by the psychosocial factors proposed by
the model has been modest, typically between 5 - 30\% (Galambos et al.,
2005; Ivarsson and Johnson, 2010); suggesting other factors are also
likely to contribute to injury occurrence.

While the psychosocial factors proposed in Williams and Andersen's
(Williams and Andersen, 1998) model have received the most research
attention, the mechanisms through which these factors are proposed to
exert their effect have remained under-investigated in the literature.
To elaborate, the model suggests that injuries are likely to occur
through either increased physiological arousal resulting in increased
muscle tension and reduced flexibility or attentional deficits caused by
increased distractibility and peripheral narrowing. However, to date,
the research has largely focused on attentional deficits (Andersen and
Williams, 1999; Rogers and Landers, 2005; Wilkerson, 2012; Swanik et
al., 2007). For example, Andersen and Williams (Andersen and Williams,
1999) measured peripheral and central vision during high and low stress
conditions and found athletes with high life event stress coupled with
low social support had greater peripheral narrowing under stressful
conditions compared to athletes with the opposing profile; these
athletes went on to sustain an increased number of injuries during the
following season. Indeed, Rodgers and Landers (Rogers and Landers, 2005)
supported Andersen and Williams's (Andersen and Williams, 1999) earlier
findings reporting that peripheral narrowing under stress mediated 8.1\%
of the relationship between negative life events and injury. However,
the remaining variance between negative life events and athletic injury
through the other proposed mechanisms, such as increased muscle tension
and reduced motor control, remains to be explored (cf.~Williams and
Andersen, 1998).

One challenge faced by researchers addressing the sports injury problem
with a psychological lens is the multifactorial nature of injury, and
the possible contribution of other non-psychological factors to the
stress response (Meeuwisse et al., 2007; Wiese-Bjornstal, 2009). For
example, a large body of research indicates that training-related stress
is also likely to be related to the stress response and injury
occurrence (Lee et al., 2017; Djaoui et al., 2017), and may account for
the unexplained variance from the psychological predictors of injury.
Appaneal and Perna (Appaneal and Perna, 2014) proposed the
biopsychosocial model of stress athletic injury and health (BMSAIH) to
serve as an extension to Williams and Andersen's (Williams and Andersen,
1998) model and to address some of these issues. To elaborate, the
BMSAIH aimed to clarify the mediating pathways between the stress
response and injury, consider other health outcomes and behavioural
factors that impact sports participation, and integrate the impact of
training on athletes' health (Appaneal and Perna, 2014). The central
tenet of the BMSAIH is that psychosocial distress (e.g., negative life
events) may act synergistically with training-related stress as a result
of high-intensity and high-volume sports training, and ``widen the
window of susceptibility'' (Appaneal and Perna, 2014, 74) to a range of
undesirable health outcomes including illness and injury. Consequently,
the BMSAIH provides a framework for future research to build on Williams
and Andersen's (Williams and Andersen, 1998) model, by including other
physiological markers of training-related stress, which together may
provide greater insight into the injury process.

Although research supporting the BMSAIH has mainly focused on the
relationship between hormonal responses to training and injury
occurrence (Perna and McDowell, 1995; Perna et al., 1997, 2003), other
research has identified additional markers of training-related stress
that are associated with an increased risk of injury; for example, heart
rate variability (Bellenger et al., 2016; Williams et al., 2017),
postural stability (Romero-Franco et al., 2014) and muscle stiffness
(Pruyn et al., 2015). However, these markers are often studied in
isolation without an assessment of the psychosocial factors that are
known to contribute to injury, thereby limiting our understanding of how
psychosocially and physiologically derived stress may contribute
synergistically to injury occurrence. Recently, Bittencourt et
al.~(Bittencourt et al., 2016) suggested that to better understand the
multifactorial nature of sports injuries, research needs to move away
from studying risk factors in isolation and instead adopt a complex
systems approach to injury. Such an approach posits that injury may
arise from a complex ``web of determinants'' (Bittencourt et al., 2016,
3), where different factors interact in unpredictable and unplanned
ways, but result in a global outcome pattern of either adaptation or
injury.

A challenge when adopting a complex systems approach is using an
appropriate analysis technique that is able to capture the uncertainty
and complexity of the relationships between different variables. One
technique that provides a solution to this challenge is Bayesian network
(BN) modelling. BN's allow the construction of graphical probabilistic
models using the underlying structure that connects different variables
(nodes in the network) (Scutari and Denis, 2014). The learned structure
can be used for inference by obtaining the posterior probabilities of a
particular node for a given query (e.g., if the value of Node A is x and
the value of Node B is y, what is the probability Node C being value
z?). Furthermore, BN's do not distinguish between dependent and
independent variables as they are a form of unsupervised learning, which
is a strength over regression or structure equation models when the
underlying relationship in the network may not be known (Olmedilla et
al., 2018).

To summarise, despite offering a possible framework to build on the
research stemming from Williams and Andersen's (Williams and Andersen,
1998) model, there remains and opportunity to explore other
physiological stress-related markers proposed by the BMSAIH, in addition
to the already well-established psychological characteristics known to
be related to injury (Appaneal and Perna, 2014). Furthermore, research
has typically not captured changes in both psychosocial factors and
stress-related physiological markers that may occur between initial
measurement and injury occurrence. Given the exploratory nature of the
current study the following objectives were defined:

\begin{itemize}
\tightlist
\item
  Identify suitable markers of stress that can be easily captured in a
  large cohort of athletes in a field based setting.
\item
  Capture the markers of stress and injury occurrence using a
  prospective, repeated measures design.
\item
  Explore and evaluate the relationships between the markers of stress
  and injury using Bayesian network modelling.
\end{itemize}

\hypertarget{methods}{%
\section*{Methods}\label{methods}}
\addcontentsline{toc}{section}{Methods}

\hypertarget{participants}{%
\subsection*{Participants}\label{participants}}
\addcontentsline{toc}{subsection}{Participants}

A total of 351 athletes (male: \emph{n} = 231, female: \emph{n} = 120)
were initially recruited for the study. Participants had an average age
of 22.0 \(\pm\) 7.0 years and represented a range of team (football,
rugby, netball, cricket, lacrosse, basketball and field hockey) and
individual sports (athletics, tennis, weightlifting, gymnastics, judo,
swimming and golf from a British University and local sports clubs
(Table \ref{tab:table1}). Participants self-rated competitive level
ranged from recreational to international standard. A total of 126
(49.03\%) participants had sustained an injury in the 12 months prior to
the start of the study (male: \emph{n} = 114 {[}49\%{]}, female;
\emph{n} = 48 {[}40\%{]}). At the start of the study, all participants
were injury free (no modifications to their usual training routine due
to a sport related medical problem for a minimum of four weeks).
Participants were engaged in training for their respective sports for at
least five hours per week. Ethical approval was obtained from the
University ethics committee prior to the start of the study and all
participants provided informed consent.

\begin{table}[H]

\caption{\label{tab:table1}Participant characteristics.}
\centering
\begin{tabular}[t]{c|c|c}
\hline
\textbf{ } & \textbf{Female (n = 120)} & \textbf{Male (n = 231)}\\
\hline
\multicolumn{3}{l}{\textbf{Demographics M (SD)}}\\
\hline
\hspace{1em}Age (yrs) & 26.0 (11.3) & 20.2 (1.8)\\
\hline
\hspace{1em}Height (cm) & 167.4 (7.6) & 177.8 (7.8)\\
\hline
\hspace{1em}Body mass (kg) & 67.0 (9.5) & 82.0 (14.6)\\
\hline
\hspace{1em}Hours per week training & 8.5 (4.5) & 11.2 (8.8)\\
\hline
\multicolumn{3}{l}{\textbf{Current competitive level n (\%)}}\\
\hline
\hspace{1em}Recreational & 3 (4) & 7 (4)\\
\hline
\hspace{1em}University & 45 (56) & 141 (80)\\
\hline
\hspace{1em}National/International & 33 (41) & 28 (16)\\
\hline
\end{tabular}
\end{table}

\hypertarget{measures}{%
\subsection*{Measures}\label{measures}}
\addcontentsline{toc}{subsection}{Measures}

\hypertarget{major-life-events}{%
\subsubsection*{Major life events}\label{major-life-events}}
\addcontentsline{toc}{subsubsection}{Major life events}

A modified version of the Life Events Survey for Collegiate Athletes
(LESCA) was used to measure participants' history of life event stress
(Petrie, 1992). Modifications were made to the LESCA to ensure the
suitability of the items for the study population (S1 Table). The LESCA
comprises 69 items that reflect possible life events that participants
may have experienced. Example items include, ``Major change in the
frequency (increased or decreased) of social activities due to
participation in sport'', ``Major change in the amount (more or less) of
academic activity (homework, class time, etc)'' and ``Major change in
level of athletic performance in actual competition (better or worse)''.
Participants were asked to rate the perceived impact of each life event
they had experienced within the last 12 months on an 8-point Likert
scale anchored at -4 (\(extremely\;negative\)) and +4
(\(extremely\;positive\)). Negative and positive life event scores were
calculated by summing the negative and positive scores respectively. A
score for total life events was calculated by summing the absolute
values for both negative and positive events. Petrie (1992) (Petrie,
1992) reported test-retest reliabilities at 1-week and 8-weeks with
values ranging from .76 to .84 (p \textless{} .001) and .48 to .72 (p
\textless{} .001) respectively. Petrie also provided evidence of
discriminant, convergent and predictive validity. The LESCA is the most
widely used measure of major life events for athletes in the sports
injury literature. For this study, Composite Reliability (Fornell and
Larcker, 1981) was preferred to Cronbach's alpha as it does not assume
parallelity (i.e., all factor loadings are constrained to be equal, and
all error variances are constrained to be equal) and instead takes into
consideration the varying factor loadings of the items in the
questionnaire. The composite reliability for the LESCA in this study was
0.84.

\hypertarget{reinforcement-sensitivity-theory-personality-questionnaire}{%
\subsubsection*{Reinforcement Sensitivity Theory Personality
Questionnaire}\label{reinforcement-sensitivity-theory-personality-questionnaire}}
\addcontentsline{toc}{subsubsection}{Reinforcement Sensitivity Theory
Personality Questionnaire}

A revised version of the Reinforcement Sensitivity Theory Personality
Questionnaire (RST-PQ) was used to measure motivation, emotion,
personality and their relevance to psychopathology (Corr and Cooper,
2016). The revised version of the RST-PQ comprises 51 statements that
measure three major systems: Fight-Flight-Freeze System (FFFS; e.g., ``I
am the sort of person who easily freezes-up when scared''), Behavioural
Inhibition System (BIS; e.g., ``When trying to make a decision, I find
myself constantly chewing it over'') and four Behavioural Approach
System (BAS) factors; Reward Interest (e.g., ``I regularly try new
activities just to see if I enjoy them''), Goal Drive Persistence (e.g.,
``I am very persistent in achieving my goals''), Reward Reactivity
(e.g., ``I get a special thrill when I am praised for something I've
done well'') and Impulsivity (e.g., ``I find myself doing things on the
spur of the moment''). Participants rated each item on a scale from 1
(\(not\; at\; all\)) to 4 (\(highly\)) to reflect how well each
statement described their personality in general. The responses to items
associated with each subscale (FFFS, BIS, RI, GDP, RR and I) were summed
to give a total score that was subsequently used for further analysis.
The composite reliabilities for each subscale were; BIS = 0.92, FFFS =
0.77, GDP = 0.87, I = 0.71, RI = 0.77, RR = 0.81. Further details
regarding the revised RST are in S1 Appendix.

\hypertarget{heart-rate-variability}{%
\subsubsection*{Heart rate variability}\label{heart-rate-variability}}
\addcontentsline{toc}{subsubsection}{Heart rate variability}

A Polar V800 heart rate monitor (HRM) and Polar H7 Bluetooth chest strap
(Polar OY, Finland) was used to collect inter-beat interval (IBI) data.
IBI recordings using the Polar V800 are highly comparable (ICC = 1.00)
with ECG recordings (Giles et al., 2016), which are considered the gold
standard for assessing HRV. In addition, HRV indices calculated from IBI
and ECG data have shown a strong correlation (r = .99) in athletes
(Caminal et al., 2018) and under spontaneous breathing conditions (Plews
et al., 2017).

\hypertarget{musculoskeletal-properties}{%
\subsubsection*{Musculoskeletal
properties}\label{musculoskeletal-properties}}
\addcontentsline{toc}{subsubsection}{Musculoskeletal properties}

A handheld myometer (MyotonPRO, Myoton AS, Tallinn, Estonia) was used to
measure muscle stiffness. The MyotonPRO is a non-invasive, handheld
device that applies a mechanical impulse of 0.40 N for 0.15 ms
perpendicular to the surface of the skin. The impulse causes natural
damped oscillations in the tissue, which are recorded by a three-axis
digital accelerometer sensor in the device. The raw oscillation signal
is then processed, and the stiffness parameter is calculated
(Agyapong-Badu et al., 2016). The MyotonPRO has previously been reported
to be a reliable and valid tool for the measurement of in-vivo tissue
stiffness properties (Chuang et al., 2013; Pruyn et al., 2016; Nair et
al., 2014), and has demonstrated good internal consistency (coefficient
of variation \textless{} 1.4\%) over sets of 10 repetitions (Aird et
al., 2012).

\hypertarget{postural-stability}{%
\subsubsection*{Postural stability}\label{postural-stability}}
\addcontentsline{toc}{subsubsection}{Postural stability}

Postural stability was assessed with a modified version of the balance
error scoring system (mBESS) based on the protocol recommended by Hunt
et al. (2009). In total, each trial of the mBESS was performed without
shoes (McCrory et al., 2013) and included six stances in the following
order; dominant leg (DL; standing on the dominant foot with the
non-dominant foot at approximately 30-degrees of hip flexion and
45-degrees of knee flexion), non-dominant leg (NDL; standing on the
non-dominant foot with the dominant foot at approximately 30-degrees of
hip flexion and 45-degrees of knee flexion) and tandem leg stance (TS;
standing heel-to-toe with the non-dominant foot behind the dominant) on
firm and foam (Alcan airex AG, Sins, Switzerland) surfaces respectively
(Fig \ref{fig:fig2}). To determine leg dominance, participants were
asked their preferred leg to kick a ball to a target, and the chosen
limb was labelled as dominant (cf.~Cingel et al., 2017). Participants
were asked to maintain each stance for a total of 20-seconds.
Participants hands were placed on hips at the level of the iliac crests.
A Sony DSC-RX10 video camera (Sony Europe Limited, Surrey, United
Kingdom) was used to record each participants performance during the
mBESS.

\begin{figure}[!h]
\caption{{\bf mBESS positions (A-F).}
Top row, firm surface. Bottom row, foam surface. Left column, dominant leg stance. Middle column, non-dominant leg stance. Right column, Tandem leg stance.}
\label{fig:fig2}
\end{figure}

The error identification criteria from the original BESS protocol was
used by the lead researcher who scored all the BESS trials. One error
was recorded if any of the following movements were observed during each
trial: a) lifting hands off iliac crests; b) opening eyes; c) stepping,
stumbling, or falling; d) moving the thigh into more than 30 degrees of
flexion or abduction; e) lifting the forefoot or heel; and f) remaining
out of the testing position for more than 5-seconds (Riemann et al.,
1999). A maximum score of 10 errors was possible for each stance.
Multiple errors occurring simultaneously were recorded as one error. A
participant was given the maximum score of 10 if they remained out of
the stance position for more than 5-seconds. To calculate limb
asymmetry, the DL and NDL leg score was calculated by summing the DL and
NDL errors respectively. A total score was calculated by summing the
total number of errors recorded on all stances (DL, NLD and TS, on foam
and firm surfaces). To assess the intra-rater reliability, a single
measurement, absolute agreement, two-way mixed effects model for the
intraclass correlation (ICC; Koo and Li, 2016) was used on a sample of
40 participants from the first time point. The test-retest scoring of
BESS resulted in a ``good'' to ``excellent'' ICC score (ICC = 0.93, 95\%
confidence interval = 0.88 - 0.96), indicating the scoring was reliable.

\hypertarget{injury}{%
\subsubsection*{Injury}\label{injury}}
\addcontentsline{toc}{subsubsection}{Injury}

Participants self-reported any injuries they sustained at each data
collection during the study period. An injury was defined as any sports
related medical problem causing the athlete to miss or modify their
usual training routine for at least 24 hours (Fuller et al., 2006, 2007;
Timpka et al., 2014). Minor scrapes and bruises that may require certain
modifications (e.g., strapping or taping) but did not limit continued
participation were not considered injuries (cf.~Appaneal et al., 2009).
Injury status (did / did not sustain an injury) served as the main
outcome measure.

\hypertarget{procedure}{%
\subsection*{Procedure}\label{procedure}}
\addcontentsline{toc}{subsection}{Procedure}

At the start of the UK academic year (September 2016 and 2017), coaches
of sports teams at a British University and local sports clubs were
contacted and informed about the study. With the coaches' permission,
the lead researcher attended training sessions to inform the athletes
about the overall purpose of the study and the requirements of
participation. To be eligible for the study athletes had to be injury
free (no modifications to their usual training routine due to a sport
related medical problem for a minimum of four weeks) and training a
minimum of five hours per week. Athletes who met the criteria and
volunteered to take part in the study were invited to attend scheduled
testing sessions. A repeated measures prospective cohort design was used
to assess athletes' major life events, stress-related physiological
markers and injury status over two consecutive twelve-month periods
between September 2016 and September 2018. Each participant was asked to
attend a total of 3 data collections over a twelve-month period, with
each data collection separated by a four-month interval (Fig
\ref{fig:fig3}). Participants provided informed consent before data
collection commenced.

\begin{figure}[!h]
\caption{{\bf Study design.}
For each time point (T), each box contains the number of participants who completed the data collection (N), the measures used for data collection and the approximate date of the data collection.}
\label{fig:fig3}
\end{figure}

For the first three data collections (T1, T2 and T3), participants
followed the same protocol in a specific order (Fig \ref{fig:fig4}). To
ensure all measures could be collected within a viable time-frame,
participants were separated into two groups. The first group completed
all computer-based measures followed by all physical measurements,
whereas the second group completed all physical measurements followed by
computer-based measures. Participants were randomly assigned to one of
the two groups and remained in those groups across all time points.

\begin{figure}[!h]
\caption{{\bf Session protocol.}
Outline of the protocol for each data collection.}
\label{fig:fig4}
\end{figure}

\hypertarget{questionnaires}{%
\subsubsection*{Questionnaires}\label{questionnaires}}
\addcontentsline{toc}{subsubsection}{Questionnaires}

The questionnaires, which included demographic information, the LESCA,
RST-PQ (T1, T2, T3) and injury status (T2, T3, T4) were completed
on-line (SurveyMonkey Inc., USA, www.surveymonkey.com). The instructions
for the LESCA were modified at T2 and T3 so that participants reported
major life events that had occurred since the previous testing session.
For injury reporting, participants were asked to record any injuries
that they had sustained since the last data collection. The data were
downloaded from surverymonkey.com and imported into R (R Core Team,
2019) for analysis purposes.

\hypertarget{hrv}{%
\subsubsection*{HRV}\label{hrv}}
\addcontentsline{toc}{subsubsection}{HRV}

To minimise potential distractions, participants were directed to a
designated quiet area in the laboratory where IBI data were recorded.
Participants were instructed to turn off their mobile devices to avoid
any interference with the Bluetooth sensor. Each chest strap was
dampened with water and adjusted so it fitted tightly but comfortably,
as outlined by Polar's guidelines. Participants were seated and asked to
remain as still as possible for the duration of the recording. No
attempt was made to control participants respiratory frequency or tidal
volume (Denver et al., 2007). Inter-beat interval (IBI) data was
collected for 10-minutes at a sampling frequency of 1000 Hz.

Raw, unfiltered IBI recordings were exported from the Polar Flow web
service as a space delimited .txt file and imported into R (R Core Team,
2019) where the \emph{RHRV} package ({\textbf{???}}) was used to
calculate HRV indices. Raw IBI data was filtered using an adaptive
threshold filter, and the first 3-minutes and last 2-minutes of each
recording were discarded, leaving a 5-minutes window that was used to
calculate the root mean square of successive differences (RMSSD) in RR
intervals following the recommendations for short term IBI recordings
(Laborde et al., 2017; Malik et al., 1996). RMSSD was calculated as:

\begin{equation} 
  \overline{RR} = \frac{1}{N} \sum_{i=1}^{n} RR_i
\end{equation}

Where N is the length of the time series, and \(RR_i\) the RR interval
between beats \(i\) and \(i-1\), where each beat position corresponds to
the beat detection instant.

\hypertarget{muscle-stiffness}{%
\subsubsection*{Muscle stiffness}\label{muscle-stiffness}}
\addcontentsline{toc}{subsubsection}{Muscle stiffness}

To assess muscle stiffness, participants lay horizontally on a massage
bed and four testing sites were identified on each lower limb. The
muscle belly of the rectus femoris (RF), biceps femoris (BF), medial
gastrocnemius (MG) and lateral gastrocnemius (LG) sites were identified
using a visual-palpatory technique to determine the exact location of
each site (Chuang et al., 2012). The visual-palpatory technique required
the participant to contract the target muscle to aid the lead researcher
to visually identify the muscle. The participant was then asked to relax
the muscle and the muscle was palpated to locate the muscle belly. A
skin safe pen (Viscot all skin marker pen, Viscot Medical LLC, NJ) was
used to mark the testing site in the centre of the muscle belly.

After the eight testing sites had been identified, the testing end of
the MyotonPRO (diameter = 3 mm) was positioned perpendicular to the skin
on the testing site. A constant pre-load of 0.18 N was applied for
initial compression of subcutaneous tissues. The device was programmed
to deliver five consecutive impulses, separated by a one second interval
(Morgan et al., 2018). For each impulse, the device computed stiffness
values, with the median of the five values being saved by the device for
further analysis. In accordance with Myoton.com, a set of five
measurements with a coefficient of variation (CV) of less than 3\% was
accepted. Sets of measurements above 3 \% were measured again to ensure
the reliability of the data. The CV was calculated in real time by the
device after each set of measurements. Measurements saved on the device
were uploaded to a computer using MyotonPRO software and imported in R
(R Core Team, 2019) for further analysis. For each participant, the sum
of all eight testing sites was calculated to provide a total lower
extremity stiffness score and was used for further analysis.

\hypertarget{postural-stability-1}{%
\subsubsection*{Postural stability}\label{postural-stability-1}}
\addcontentsline{toc}{subsubsection}{Postural stability}

Instructions for the mBESS were read to each participant and a
demonstration of the positions was provided by the research assistant.
For each position, participants were instructed to close their eyes,
rest their hands on their iliac crests and remain as still as possible
for 20-seconds. Participants were instructed to return to the testing
position as quickly as possible if they lost their balance. The video
recording was started prior to the first stance position and stopped
after all stances had been completed. Each completed mBESS protocol took
approximately 4 minutes. Only one trial was performed to avoid
familiarisation effects across the repeated measurement (cf.~Valovich et
al., 2003). The video recordings for each participant were imported from
the recording equipment (Sony DSC-RX10) and the lead researcher scored
each trial using the error identification criteria.

\hypertarget{data-analysis}{%
\subsection*{Data Analysis}\label{data-analysis}}
\addcontentsline{toc}{subsection}{Data Analysis}

Two Bayesian Networks (BN) were used to explore the relationships
between the psychological measures, physiological markers of stress and
sports injury. A BN is a graphical representation of a joint probability
distribution among a set of random variables, and provides a statistical
model describing the dependencies and conditional independences from
empirical data in a visually appealing way (Scutari and Denis, 2014). A
BN consists of arcs and nodes that together are formally known as a
directed acyclic graph (DAG), where a node is termed a parent of a child
if there is an arc directed from the former to the latter (Fig
\ref{fig:fig5}; Pearl, 1988). However, the direction of the arc does not
necessarily imply causation, and the relationship between variables are
often described as probabilistic instead of casual (Scutari and Denis,
2014). The information within a node can be either continuous or
discrete, and a complete network can contain both continuous and
discrete nodes; however, discrete networks are the most commonly used
form of BN (Chen and Pollino, 2012). In discrete networks, conditional
probabilities for each child node are allocated for each combination of
the possible states in their parent nodes and can be used to assess the
strength of a dependency in the network.

\begin{figure}[H]
\caption{{\bf Example network.}
A simple discrete network contain nodes, possible states of the nodes and the arcs connecting nodes.}
\label{fig:fig5}
\end{figure}

In order to use discrete networks, continuous variables must first be
split into categorical levels. When there are a large number of
variables in the network, limiting the number of levels has the benefit
of producing a network that is more parsimonious in terms of parameters.
For example, a network with 10 variables each with two levels has 100
(10\^{}2) possible parameter combinations, however the same network with
three levels has 1000 (10\^{}3) possible parameter combinations, the
latter being significantly more computationally expensive. Using a
larger number of splits in the data also comes at a cost of reducing the
statistical power in detecting probabilistic associations, and reduces
the precision of parameter estimates for the probabilistic associations
that are detected because it reduces the sample-size-to-parameters ratio
(Scutari and Denis, 2014).

Learning the structure of the network is an important step in BN
modelling. The structure of a network can be constructed using expert
knowledge and/or data-driven algorithm techniques (e.g., search and
score, such as hill climbing and gradient descent algorithms; Scutari \&
Denis, 2014). The learned structure can then be used for inference by
querying the network\footnote{The term ``query'' in relation to Bayesian
  Networks stems from Pearl's expert systems theory (1988). A query can
  be submitted to an expert (in this case, the network is the expert) to
  get an opinion, the expert then updates the querier's beliefs
  accordingly. Widely used texts on Bayesian Network analysis (Koller
  and Friedman, 2009) have widely adopted the terminology in favour of
  that used in traditional statistics.} and obtaining the posterior
probabilities of a particular node for a given query. The posterior
distribution can be obtained by \(Pr(X|E,B) = Pr(X|E,G,\Theta)\), where
the learned network \(B\) with structure \(G\) and parameters
\(\Theta\), are investigated with new evidence \(E\) using the
information in \(B\) (Scutari and Denis, 2014). In the example network
presented in Fig \ref{fig:fig5}, new values assigned to each of the
parent nodes (e.g., both set to ``Low'') could be used to investigate
what effect the new information has on the state of the child node
(conditional probability of a particular state of the child node). In a
more complex network containing many nodes, the outcome of a particular
node can be assessed conditional on the states of any subset of nodes in
the network. BNs therefore provide a unique and versatile approach to
modelling a set of variables to uncover dependency structures within the
data.

BNs have recently been used in the sport psychology literature
(Olmedilla et al., 2018; Fuster-Parra et al., 2017) and offer several
benefits over traditional statistical analysis. For example, predictions
can be made about any variable in the network, rather than there being a
distinction between dependent and independent variables in the data,
such as in linear regression models that are often used within the sport
psychology literature (Olmedilla et al., 2018; Bittencourt et al.,
2016). Furthermore, the structure of a network can be obtained from both
empirical data \emph{and} prior knowledge about the area of study; the
latter being particularly useful when there are a large number of
variables in the network, or only a small number of observations are
available in the data (Xiao-xuan et al., 2008). In such instances, a
purely data driven approach to learning the network would be
time-consuming due to the large parameter space, and inefficient at
identifying an approximation of the true network structure. Prior
knowledge about dependencies between variables can therefore be included
in the network structure, while still allowing a data driven approach
for unknown dependencies, to improve the overall computation of the
network structure (Heckerman et al., 1995; Xu et al., 2015). The
following sections detail the steps taken in the current study to
firstly prepare the data for the network, and then obtain the structure
of the network that was used for inference.

\hypertarget{data-preparation}{%
\subsubsection*{Data Preparation}\label{data-preparation}}
\addcontentsline{toc}{subsubsection}{Data Preparation}

Of the 351 participants that were initially recruited for the study, 94
only completed the first time point, and therefore had to be removed
from the study as no injury information was obtained for these
participants following the first time point. To prepare the data for the
BN, missing values in the dataset were first imputed. Out of the 650
total measurements across all time points in the current study, there
were 31 (4.64\%) missing muscle stiffness measurements and 70 (10.48\%)
missing heart rate recordings. The missing data were due to technical
faults in the data collection equipment and were considered to be
missing completely at random. A missing rate of 15-20\% has been
reported to be common in psychological studies, and several techniques
are available to handle missing values (Enders, 2003; Lang et al.,
2013). In the current study, the \emph{caret} package (Kuhn et al.,
2008) was used to impute the missing values. A bagged tree model using
all of the non-missing data was first generated and then used to predict
each missing value in the dataset. The bagged tree method is a reliable
and accurate method for imputing missing values in data and is superior
to other commonly used methods such a median imputation (Kuhn et al.,
2008).

A median split technique was used to discretise the data used in the
network into ``Low'' and ``High'' levels. All variables apart from
negative and total life events were approximately normally distributed
(based on visual inspection, see supplementary table \ldots) and
required no further transformation prior to the median split. For the
LESCA questionnaire data, a cumulative total of the current, and
previous time points was calculated at each time point to account for
the potential continuing effect of the life events experienced by
athletes over time. Given the limited support for a relationship between
positive life events and injury (Williams and Andersen, 2007), only
negative and total life events were included in the network. Cumulative
negative, and cumulative total life event scores at each time point were
first log scaled so distributions were approximately normal, and then
binarised using the median at each time point (nlelg and tlelg
respectively). In addition to the log scaled cumulative values, an
untransformed NLE score from the first time point was included as an
additional variable based on previous literature that indicates this
variable should have a strong relationship with injury outcome (Ivarsson
et al., 2017).

\hypertarget{network-structure}{%
\subsubsection*{Network structure}\label{network-structure}}
\addcontentsline{toc}{subsubsection}{Network structure}

To obtain the network structure, several steps were taken to ensure that
both a theoretically realistic network, and a network that was an
appropriate fit to the collected data, was used for inference. Prior
knowledge about the network structure was included by providing a list
of arcs that are always \emph{restricted} from being in the network
(blacklist), and a list of arcs that are always \emph{included} in the
network (whitelist). Additionally, there are several scoring functions
such as Bayesian Information Criteria (BIC) and Bayesian Dirichlet
equivalent uniform (BDeu) that can be used to compare network structures
with certain nodes and arcs included or excluded (Scutari and Denis,
2014). To account for the repeated measures design and to maximise the
use of the data, pairs of complete cases (e.g., participants who
completed T1 + T2, and T2 + T3) were used in a two-time Bayesian network
(2TBN) structure (Murphy, 2002). In the 2TBN, variables measured T2
could depend on variables measured at T1 (e.g., T1 \(\rightarrow\) T2)
and variables measured at T3 could depend on variables measured at T2
(e.g., T2 \(\rightarrow\) T3). However, arcs were blacklisted between T2
\(\rightarrow\) T1 and T3 \(\rightarrow\) T2 to preserve the order in
which data was collected. Variables were separated into two groups;
``explanatory'', for variables that did not change during the study
(e.g., gender), or ``independent'', for variables that were measured at
each time point and could vary during the study. Independent variable
names were suffixed with \_1 for time point T, and \_2 for time point
T+1 (e.g., T1\_1 \(\rightarrow\) T2\_2 and T2\_1 \(\rightarrow\) T3\_2).
Formatting the data in this way meant participants who completed T1 and
T2, but did not complete T3, could still be included in the analysis.
Table \ref{tab:table2} provides an example of the formatted data.
Participants 1 and 3 have complete data, and therefore have two rows of
data each representing variables from T1 \(\rightarrow\) T2 and T2
\(\rightarrow\) T3, respectively. Participant 2 did not complete the
final data collection at T3 and therefore only has one row of data
representing the variables collected at T1 and T2. In addition to the
blacklisted arcs between T2 \(\rightarrow\) T1 and T3 \(\rightarrow\)
T2, the direction of arcs was restricted between independent variables
and explanatory variables (e.g., independent \(\rightarrow\)
explanatory); however, arcs were not restricted between explanatory
\(\rightarrow\) independent variables. Finally, arc direction was
restricted between specific nodes within the explanatory variables. Arcs
from clevel \(\rightarrow\) gender, nlebase \(\rightarrow\) gender and
nlebase \(\rightarrow\) ind\_team were included in the blacklist, as
arcs in these directions did not make logical sense. All subsequent
models used the same blacklist.

\begin{table}[H]

\caption{\label{tab:table2}Example of the data arrangement used for the network.}
\centering
\begin{tabular}[t]{l|l|l|l}
\hline
\textbf{Participant} & \textbf{X\_1} & \textbf{ } & \textbf{X\_2}\\
\hline
1 & T1 & -> & T2\\
\hline
1 & T2 & -> & T3\\
\hline
2 & T1 & -> & T2\\
\hline
3 & T1 & -> & T2\\
\hline
3 & T2 & -> & T3\\
\hline
\end{tabular}
\end{table}

\hypertarget{preliminary-network-structures}{%
\subsubsection*{Preliminary network
structures}\label{preliminary-network-structures}}
\addcontentsline{toc}{subsubsection}{Preliminary network structures}

Prior to the final network structure presented in the results section,
several structures with different combinations of variables were
explred. Networks were learned using a Tabu search algorithm (Russell
and Norvig, 2009) and BIC was used to compare different models. A higher
BIC value indicates the structure of a DAG is a better fit to the
observed data (Scutari and Denis, 2014). BIC values for each combination
of variables of interest are reported as the combination of variables
with the highest BIC value, followed by the relative scores of the other
variables in the model.

Initially, both negative life events and total life events were included
in the network structure, however, the network score was improved when
only nlelg or tlelg was included (highest BIC value = nleleg, BIC values
relative to nlelg; tlelg only = -85.13, tlelg and nlelg = -219.14).
Additionally, despite strong evidence in the literature that both
negative and total life event stress are related to injury occurrence
(Williams and Andersen, 2007), network structures learned using the Tabu
search algorithm failed to identify a relationship between NLE and
injury or TLE and injury in the data. Given that nlelg provided the
highest network score, and there is a stronger relationship between
negative life events and injury in the literature, an arc was
whitelisted between nlelg\_1 and injured\_1 and nlelg\_2 and injured\_2
in the final network structure. Total life event score was not included
in the final structure.

The subscales representing the BAS (RR, RI, GDP and I) showed limited
connection to other variables in the network. Therefore, several models
were run with each scale individually to find the scale that resulted in
the highest BIC value (values are shown relative to the highest value).
RI provided the highest BIC value, compared to RR (-13.13), GDP (-15.06)
and I (-15.2). Including all the variables (RR, RI, GDP and I) resulted
in a significantly lower score -894.37) indicating that including all
the variables was not beneficial to the model structure and did not
offset the cost of the additional parameters. Therefore, only RI was
included in the final structure.

Finally, both total score and asymmetry were included in the initial
network. However, visual inspection of the network revealed no arcs
between bal\_asym\_1 or bal\_asym\_2 and any other node in the network.
Therefore, balance asymmetry was removed from the final network
structure. To summarise, Table \ref{tab:table3} includes the variables
that were included in the final network structure.

\begin{table}[H]
\begin{adjustwidth}{-2.25in}{0in}
\caption{\label{tab:table3}Variables included in the final network structure.}
\centering
\begin{tabular}{l|l|l|l}
\hline
{\bf Variable} & {\bf Definition} & {\bf State} 1 & {\bf State 2}\\
\hline
clevel & Current competitive level & Club\_university\_county & National\_international\\
\hline
gender & Gender of the participant & Female & Male\\
\hline
hours & Number of hours spent training per week & 0-9 (Low) & >9-35 (High)\\
\hline
ind\_team & Participate in an individual or team based sport & Individual & Team\\
\hline
pi & Sustained previous injury & No Injury & Injury\\
\hline
nlebase & Untransformed NLE at TP 1 & 0-13 (Low) & >13-93 (High)\\
\hline
FFFS & Fight-Flight-Freeze System & 8-16 (Low) & >16-30 (High)\\
\hline
BIS & Behavioural Inhibition System & 17-38 (Low) & >38-68 (High)\\
\hline
RI & Reward Interest & 4-10 (Low) & >10-16 (High)\\
\hline
stiffness & Sum of all stiffness locations & 1543-2330 (Low) & >2330-4518 (High)\\
\hline
rmssd & Root mean squared difference of successive RR intervals & 2.03-4.02 (Low) & >4.02-5.94 (High)\\
\hline
balance & Total balance score & 5-15 (Low) & >15-46 (High)\\
\hline
nlelg\_1 & Log NLE at TP 1 & 0-2.64 (Low) & >2.64-4.54 (High)\\
\hline
nlelg\_2 & Log NLE at TP 2 & 0-3.04 (Low) & >3.04-5.19 (High)\\
\hline
nlelg\_3 & Log NLE at TP 3 & 0-3.18 (Low) & >3.18-4.79 (High)\\
\hline
\end{tabular}
\end{adjustwidth}
\end{table}

Preliminary network structures also revealed strong dependencies between
the same variables at subsequent time points. For example, the
probability that stiffness\_1 and stiffness\_2 were both ``High'', or
both ``Low'' was approximately 80\%. Including the arcs between the same
variables from X\_1 \(\rightarrow\) X\_2 did not provide any
theoretically meaningful information to the network structure as the
majority of participants would be either be in a ``Low'' or ``High''
state for each pair of variables in the network. To more appropriately
assess changes \emph{within} variables over time, a second BN was
investigated by modelling the differences between variables at different
time points. The use of differential equations to model changes in
variables over time is a common procedure in BN analysis when there are
repeated measurements in the data (Scutari et al., 2017). To obtain the
structure, variables suffixed with \_1 were subtracted from variables
suffixed with \_2 to calculate the difference between variables measured
at time points T1 \(\rightarrow\) T2 and T2 \(\rightarrow\) T3.
Independent variables were then standardized to allow relative changes
between variables to be compared. The ``injured'' variable was also
modified to represent whether a participant had sustained an injury at
any point over the duration of the study or were healthy for the
duration of the study. The result was a network that explicitly modelled
the \emph{amount} of change within variables between time points, as
opposed to the first network that would only have captured changes when
the median threshold was crossed from ``Low'' to ``High''. Identical
blacklists to the first network were used for arcs between independent
and explanatory variables. The nlebase variable was also dropped from
the list of explanatory variables to allow the \emph{changes} in
negative life events to be the only life event variable in the network.

To obtain the final networks, the appropriate blacklist and whitelists
were provided and a Tabu search algorithm identified the remaining
structure of the network. The final network structure was obtained by
averaging 1000 bootstrapped models (Efron and Tibshirani, 1994) to
reduce the impact of locally optimal, but globally suboptimal network
learning, and to obtain a more robust model (Olmedilla et al., 2018).
Arcs that were present in at least 30\% of the models were included in
the averaged model. The strength of each arc was determined by the
percentage of models that the arc was included in, independent of the
arc's direction. An arc strength of 1 indicates that the arc is always
present in the network, with the value decreasing as arcs are found in
fewer networks. In the respective study arcs above 0.5 were considered
``significant'' with arcs below 0.5 and above 0.3 ``non-significant''
(Scutari and Nagarajan, 2013). Arcs below 0.3 were not included in the
model. The full table of arc strengths for the first and second network
are available in S2 Table and S3 Table respectively.

\hypertarget{network-inference}{%
\subsubsection*{Network Inference}\label{network-inference}}
\addcontentsline{toc}{subsubsection}{Network Inference}

Conditional probability queries (CPQ) were used to perform inference on
both network structures. To conduct a CPQ, the joint probability
distribution of the nodes was modified to include a new piece of
evidence. The query allows the odds of a particular node state (e.g.,
injured\_1 = ``injured'') to be calculated based on the new evidence.
CPQs were performed using a likelihood weighting approach; a form of
importance sampling where random observations are generated from the
probability distribution in such a way that all observations match the
evidence given in the query. The algorithm then re-weights each
observation based on the evidence when computing the conditional
probability for the query (Scutari and Denis, 2014). Inference was first
performed on arcs that had a strength greater than 0.50 between the
explanatory variables and independent variables and between different
independent variables in the network. Of particular interest in the
current study were the variables that were connected to ``injured''
nodes. To examine the variables that were associated with injured nodes
in the network, the Markov blanket of ``injured\_1'' and ``injured\_2''
were examined. A Markov blanket contains all the nodes that make the
node of interest conditionally independent from the rest of the network
(Fuster-Parra et al., 2017). CPQ's were used to determine what effect
the variables in the Markov blanket of injured nodes had on the
probability of the injured node being in the ``injured'' state.

The second network contained both continuous and discrete data. To
examine dependencies between continuous variables with arc strengths
above 0.5 in the second network, random samples were generated based on
the conditional distribution of the nodes included as evidence in the
query. The samples were then extracted and examined with Bayesian linear
regression models using the \emph{brms} package (Bürkner, 2017) to
determine the relationship between nodes in the network. Similar to the
first network, the Markov blanket of the ``injured'' node was also
investigated by determining the highest probability of injury with
combinations of variables in the Markov blanket below the mean change,
at the mean change and above the mean change.

\hypertarget{refs}{}
\leavevmode\hypertarget{ref-Agyapong-Badu2016}{}%
Agyapong-Badu, S., Warner, M., Samuel, D., and Stokes, M. (2016).
Measurement of ageing effects on muscle tone and mechanical properties
of rectus femoris and biceps brachii in healthy males and females using
a novel hand-held myometric device. \emph{Archives of Gerontology and
Geriatrics} 62, 59--67.
doi:\href{https://doi.org/10.1016/j.archger.2015.09.011}{10.1016/j.archger.2015.09.011}.

\leavevmode\hypertarget{ref-Aird2012}{}%
Aird, L., Samuel, D., and Stokes, M. (2012). Quadriceps muscle tone,
elasticity and stiffness in older males: Reliability and symmetry using
the MyotonPRO. \emph{Archives of Gerontology and Geriatrics} 55,
e31--e39.
doi:\href{https://doi.org/https://doi.org/10.1016/j.archger.2012.03.005}{https://doi.org/10.1016/j.archger.2012.03.005}.

\leavevmode\hypertarget{ref-Andersen1988}{}%
Andersen, M. B., and Williams, J. M. (1988). A model of stress and
athletic injury: Prediction and prevention. \emph{Journal of Sport and
Exercise Psychology} 10, 294--306.
doi:\href{https://doi.org/10.1123/jsep.10.3.294}{10.1123/jsep.10.3.294}.

\leavevmode\hypertarget{ref-Andersen1999}{}%
Andersen, M. B., and Williams, J. M. (1999). Athletic injury,
psychosocial factors and perceptual changes during stress. \emph{Journal
of Sports Sciences} 17, 735--741.
doi:\href{https://doi.org/10.1080/026404199365597}{10.1080/026404199365597}.

\leavevmode\hypertarget{ref-Appaneal2009}{}%
Appaneal, R. N., Levine, B. R., Perna, F. M., and Roh, J. L. (2009).
Measuring postinjury depression among male and female competitive
athletes. \emph{Journal of Sport and Exercise Psychology} 31, 60--76.
Available at: \url{http://www.ncbi.nlm.nih.gov/pubmed/19325188}.

\leavevmode\hypertarget{ref-Appaneal2014}{}%
Appaneal, R. N., and Perna, F. M. (2014). ``Biopsychosocial model of
injury,'' in \emph{Encyclopedia of sport and exercise psychology}, eds.
R. C. Eklund and G. Tenenbaum (Thousand Oaks, CA: Sage), 74--77.

\leavevmode\hypertarget{ref-Bellenger2016}{}%
Bellenger, C. R., Fuller, J. T., Thomson, R. L., Davison, K., Robertson,
E. Y., and Buckley, J. D. (2016). Monitoring athletic training status
through autonomic heart rate regulation: A systematic review and
meta-analysis. \emph{Sports Medicine} 46, 1461--1486.
doi:\href{https://doi.org/10.1007/s40279-016-0484-2}{10.1007/s40279-016-0484-2}.

\leavevmode\hypertarget{ref-Bittencourt2016}{}%
Bittencourt, N. F. N., Meeuwisse, W. H., Mendonça, L. D.,
Nettel-Aguirre, A., Ocarino, J. M., and Fonseca, S. T. (2016). Complex
systems approach for sports injuries: Moving from risk factor
identification to injury pattern recognition - narrative review and new
concept. \emph{British Journal of Sports Medicine} 50, 1309--1314.
doi:\href{https://doi.org/10.1136/bjsports-2015-095850}{10.1136/bjsports-2015-095850}.

\leavevmode\hypertarget{ref-Brewer2012}{}%
Brewer, B. W. (2012). ``Psychology of sport injury rehabilitation,'' in
\emph{Handbook of sport psychology}, eds. G. Tenenbaum and R. C. Eklund
(Hoboken, NJ, USA: Wiley), 404--424.
doi:\href{https://doi.org/10.1002/9781118270011.ch18}{10.1002/9781118270011.ch18}.

\leavevmode\hypertarget{ref-Burkner2017a}{}%
Bürkner, P. C. (2017). brms: An R package for Bayesian multilevel models
using Stan. \emph{Journal of Statistical Software} 80.
doi:\href{https://doi.org/10.18637/jss.v080.i01}{10.18637/jss.v080.i01}.

\leavevmode\hypertarget{ref-Caminal2018}{}%
Caminal, P., Sola, F., Gomis, P., Guasch, E., Perera, A., Soriano, N.,
and Mont, L. (2018). Validity of the Polar V800 monitor for measuring
heart rate variability in mountain running route conditions.
\emph{European Journal of Applied Physiology} 118, 669--677.
doi:\href{https://doi.org/10.1007/s00421-018-3808-0}{10.1007/s00421-018-3808-0}.

\leavevmode\hypertarget{ref-Chen2012}{}%
Chen, S. H., and Pollino, C. A. (2012). Good practice in Bayesian
network modelling. \emph{Environmental Modelling and Software} 37,
134--145.
doi:\href{https://doi.org/10.1016/j.envsoft.2012.03.012}{10.1016/j.envsoft.2012.03.012}.

\leavevmode\hypertarget{ref-Chuang2013}{}%
Chuang, L. L., Lin, K. C., Wu, C. Y., Chang, C. W., Chen, H. C., Yin, H.
P., and Wang, L. (2013). Relative and absolute reliabilities of the
myotonometric measurements of hemiparetic arms in patients with stroke.
\emph{Archives of Physical Medicine and Rehabilitation} 94, 459--466.
doi:\href{https://doi.org/10.1016/j.apmr.2012.08.212}{10.1016/j.apmr.2012.08.212}.

\leavevmode\hypertarget{ref-Chuang2012}{}%
Chuang, L. L., Wu, C. Y., and Lin, K. C. (2012). Reliability, validity,
and responsiveness of myotonometric measurement of muscle tone,
elasticity, and stiffness in patients with stroke. \emph{Archives of
Physical Medicine and Rehabilitation} 93, 532--540.
doi:\href{https://doi.org/10.1016/j.apmr.2011.09.014}{10.1016/j.apmr.2011.09.014}.

\leavevmode\hypertarget{ref-VanCingel2017}{}%
Cingel, R. E. H. van, Hoogeboom, T. J., Melick, N. van, Meddeler, B. M.,
and Nijhuis-van der Sanden, M. W. G. (2017). How to determine leg
dominance: The agreement between self-reported and observed performance
in healthy adults. \emph{Plos One} 12, 1--9.
doi:\href{https://doi.org/10.1371/journal.pone.0189876}{10.1371/journal.pone.0189876}.

\leavevmode\hypertarget{ref-Corr2016c}{}%
Corr, P. J., and Cooper, A. J. (2016). The reinforcement sensitivity
theory of personality questionnaire (RST-PQ): Development and
validation. \emph{Psychological Assessment} 28, 1427--1440.
doi:\href{https://doi.org/10.1037/pas0000273}{10.1037/pas0000273}.

\leavevmode\hypertarget{ref-Denver2007}{}%
Denver, J. W., Reed, S. F., and Porges, S. W. (2007). Methodological
issues in the quantification of respiratory sinus arrhythmia.
\emph{Biological Psychology} 74, 286--294.
doi:\href{https://doi.org/10.1016/j.biopsycho.2005.09.005}{10.1016/j.biopsycho.2005.09.005}.

\leavevmode\hypertarget{ref-Djaoui2017}{}%
Djaoui, L., Haddad, M., Chamari, K., and Dellal, A. (2017). Monitoring
training load and fatigue in soccer players with physiological markers.
\emph{Physiology and Behavior} 181, 86--94.
doi:\href{https://doi.org/10.1016/j.physbeh.2017.09.004}{10.1016/j.physbeh.2017.09.004}.

\leavevmode\hypertarget{ref-Efron1993}{}%
Efron, B., and Tibshirani, R. J. (1994). \emph{An introduction to the
bootstrap}. New York: Chapman \& Hall Available at:
\url{https://www.crcpress.com/An-Introduction-to-the-Bootstrap/Efron-Tibshirani/p/book/9780412042317}.

\leavevmode\hypertarget{ref-Enders2003}{}%
Enders, C. K. (2003). Using the expectation maximization algorithm to
estimate coefficient alpha for scales with item-level missing data.
\emph{Psychological Methods} 8, 322--337.
doi:\href{https://doi.org/10.1037/1082-989X.8.3.322}{10.1037/1082-989X.8.3.322}.

\leavevmode\hypertarget{ref-Fornell1981}{}%
Fornell, C., and Larcker, D. F. (1981). Evaluating structural equation
models with unobservable variables and measurement error. \emph{Journal
of Marketing Research} 18, 39--50.
doi:\href{https://doi.org/10.2307/3151312}{10.2307/3151312}.

\leavevmode\hypertarget{ref-Fuller2006}{}%
Fuller, C. W., Ekstrand, J., Junge, A., Andersen, T. E., Bahr, R.,
Dvorak, J., Hägglund, M., McCrory, P., and Meeuwisse, W. H. (2006).
Consensus statement on injury definitions and data collection procedures
in studies of football (soccer) injuries. \emph{British journal of
sports medicine} 40, 193--201.
doi:\href{https://doi.org/10.1136/bjsm.2005.025270}{10.1136/bjsm.2005.025270}.

\leavevmode\hypertarget{ref-Fuller2007b}{}%
Fuller, C. W., Molloy, M. G., Bagate, C., Bahr, R., Brooks, J. H. M.,
Donson, H., Kemp, S. P. T., McCrory, P., McIntosh, A. S., Meeuwisse, W.
H., et al. (2007). Consensus statement on injury definitions and data
collection procedures for studies of injuries in rugby union.
\emph{British journal of sports medicine} 41, 328--31.
doi:\href{https://doi.org/10.1136/bjsm.2006.033282}{10.1136/bjsm.2006.033282}.

\leavevmode\hypertarget{ref-Fuster-Parra2017}{}%
Fuster-Parra, P., Vidal-Conti, J., Borràs, P. A., and Palou, P. (2017).
Bayesian networks to identify statistical dependencies. A case study of
Spanish university students' habits. \emph{Informatics for Health and
Social Care} 42, 166--179.
doi:\href{https://doi.org/10.1080/17538157.2016.1178117}{10.1080/17538157.2016.1178117}.

\leavevmode\hypertarget{ref-Galambos2005}{}%
Galambos, S. A., Terry, P. C., Moyle, G. M., and Locke, S. A. (2005).
Psychological predictors of injury among elite athletes. \emph{British
Journal of Sports Medicine} 39, 351--354.
doi:\href{https://doi.org/10.1136/bjsm.2005.018440}{10.1136/bjsm.2005.018440}.

\leavevmode\hypertarget{ref-Giles2016}{}%
Giles, D., Draper, N., and Neil, W. (2016). Validity of the Polar V800
heart rate monitor to measure RR intervals at rest. \emph{European
Journal of Applied Physiology} 116, 563--571.
doi:\href{https://doi.org/10.1007/s00421-015-3303-9}{10.1007/s00421-015-3303-9}.

\leavevmode\hypertarget{ref-Heckerman1995}{}%
Heckerman, D., Geiger, D., and Chickering, D. M. (1995). Learning
Bayesian networks: The combination of knowledge and statistical data.
\emph{Machine Learning} 20, 197--243.
doi:\href{https://doi.org/10.1023/A:1022623210503}{10.1023/A:1022623210503}.

\leavevmode\hypertarget{ref-Hughes2014}{}%
Hughes, G. (2014). A review of recent perspectives on biomechanical risk
factors associated with anterior cruciate ligament injury.
\emph{Research in Sports Medicine} 22, 193--212.
doi:\href{https://doi.org/10.1080/15438627.2014.881821}{10.1080/15438627.2014.881821}.

\leavevmode\hypertarget{ref-Hunt2009}{}%
Hunt, T. N., Ferrara, M. S., Bornstein, R. A., and Baumgartner, T. A.
(2009). The reliability of the modified balance error scoring system.
\emph{Clinical Journal of Sport Medicine} 19, 471--475.
doi:\href{https://doi.org/10.1097/JSM.0b013e3181c12c7b}{10.1097/JSM.0b013e3181c12c7b}.

\leavevmode\hypertarget{ref-Ivarsson2010}{}%
Ivarsson, A., and Johnson, U. (2010). Psychological factors as
predictors of injuries among senior soccer players. A prospective study.
\emph{Journal of Sports Science and Medicine} 9, 347--352. Available at:
\url{https://www.ncbi.nlm.nih.gov/pmc/articles/PMC3761721/}.

\leavevmode\hypertarget{ref-Ivarsson2017}{}%
Ivarsson, A., Johnson, U., Andersen, M. B., Tranaeus, U., Stenling, A.,
and Lindwall, M. (2017). Psychosocial factors and sport injuries:
Meta-analyses for prediction and prevention. \emph{Sports Medicine} 47,
353--365.
doi:\href{https://doi.org/10.1007/s40279-016-0578-x}{10.1007/s40279-016-0578-x}.

\leavevmode\hypertarget{ref-Koller2009}{}%
Koller, D., and Friedman, N. (2009). \emph{Probabilistic graphical
models: Principles and techniques - Adaptive computation and machine
learning}. Cambridge, MA: The MIT Press
doi:\href{https://doi.org/10.1017/CBO9781107415324.004}{10.1017/CBO9781107415324.004}.

\leavevmode\hypertarget{ref-Koo2016}{}%
Koo, T. K., and Li, M. Y. (2016). A guideline of selecting and reporting
intraclass correlation coefficients for reliability research.
\emph{Journal of Chiropractic Medicine} 15, 155--63.
doi:\href{https://doi.org/10.1016/j.jcm.2016.02.012}{10.1016/j.jcm.2016.02.012}.

\leavevmode\hypertarget{ref-Kuhn2008}{}%
Kuhn, M. C. from J. W., Weston, S., Williams, A., Keefer, C., and
Engelhardt, A. (2008). caret: Classification and Regression Training.
\emph{Journal of Statistical Software} 28.
doi:\href{https://doi.org/10.1053/j.sodo.2009.03.002}{10.1053/j.sodo.2009.03.002}.

\leavevmode\hypertarget{ref-Kumar2001}{}%
Kumar, S. (2001). Theories of musculoskeletal injury causation.
\emph{Ergonomics} 44, 17--47.
doi:\href{https://doi.org/10.1080/00140130120716}{10.1080/00140130120716}.

\leavevmode\hypertarget{ref-Laborde2017}{}%
Laborde, S., Mosley, E., and Thayer, J. F. (2017). Heart rate
variability and cardiac vagal tone in psychophysiological research -
Recommendations for experiment planning, data analysis, and data
reporting. \emph{Frontiers in Psychology} 8, 1--18.
doi:\href{https://doi.org/10.3389/fpsyg.2017.00213}{10.3389/fpsyg.2017.00213}.

\leavevmode\hypertarget{ref-Lang2014}{}%
Lang, K. M., Jorgensen, T. D., Moore, E. W. G., and Little, T. D.
(2013). On the joys of missing data. \emph{Journal of Pediatric
Psychology} 39, 151--162.
doi:\href{https://doi.org/10.1093/jpepsy/jst048}{10.1093/jpepsy/jst048}.

\leavevmode\hypertarget{ref-Lavallee1996}{}%
Lavallée, L., and Flint, F. (1996). The relationship of stress,
competitive anxiety, mood state, and social support to athletic injury.
\emph{Journal of Athletic Training} 31, 296--299. Available at:
\url{http://www.ncbi.nlm.nih.gov/pubmed/16558413}.

\leavevmode\hypertarget{ref-Leddy1994}{}%
Leddy, M. H., Lambert, M. J., and Ogles, B. M. (1994). Psychological
consequences of athletic injury among high-level competitors.
\emph{Research Quarterly for Exercise and Sport} 65, 347--354.
doi:\href{https://doi.org/10.1080/02701367.1994.10607639}{10.1080/02701367.1994.10607639}.

\leavevmode\hypertarget{ref-Lee2017}{}%
Lee, E. C., Fragala, M. S., Kavouras, S. A., Queen, R. M., Pryor, J. L.,
and Casa, D. J. (2017). Biomarkers in sports and exercise: Tracking
health, performance, and recovery in athletes. \emph{Journal of Strength
and Conditioning Research} 31, 2920--2937.
doi:\href{https://doi.org/10.1519/JSC.0000000000002122}{10.1519/JSC.0000000000002122}.

\leavevmode\hypertarget{ref-Maddison2005}{}%
Maddison, R., and Prapavessis, H. (2005). A psychological approach to
the prediction and prevention of athletic injury. \emph{Journal of Sport
and Exercise Psychology} 27, 289--310.
doi:\href{https://doi.org/10.1123/jsep.27.3.289}{10.1123/jsep.27.3.289}.

\leavevmode\hypertarget{ref-Malik1996}{}%
Malik, M., Camm, A. J., Bigger, J. T., Breithardt, G., Cerutti, S.,
Cohen, R. J., Coumel, P., Fallen, E. L., Kennedy, H. L., Kleiger, R. E.,
et al. (1996). Heart rate variability. Standards of measurement,
physiological interpretation, and clinical use. \emph{European Heart
Journal} 17, 354--381.
doi:\href{https://doi.org/10.1093/oxfordjournals.eurheartj.a014868}{10.1093/oxfordjournals.eurheartj.a014868}.

\leavevmode\hypertarget{ref-McCrory2013}{}%
McCrory, P., Meeuwisse, W. H., Aubry, M., Cantu, R. C., Dvořák, J.,
Echemendia, R. J., Engebretsen, L., Johnston, K., Kutcher, J. S.,
Raftery, M., et al. (2013). Consensus statement on concussion in sport:
The 4th international conference on concussion in sport, Zurich,
November 2012. \emph{Journal of Athletic Training} 48, 554--575.
doi:\href{https://doi.org/10.4085/1062-6050-48.4.05}{10.4085/1062-6050-48.4.05}.

\leavevmode\hypertarget{ref-Meeuwisse2007}{}%
Meeuwisse, W. H., Tyreman, H., Hagel, B., and Emery, C. (2007). A
dynamic model of etiology in sport injury: The recursive nature of risk
and causation. \emph{Clinical Journal of Sport Medicine} 17, 215--219.
doi:\href{https://doi.org/10.1097/JSM.0b013e3180592a48}{10.1097/JSM.0b013e3180592a48}.

\leavevmode\hypertarget{ref-Morgan2018}{}%
Morgan, G. E., Martin, R., Williams, L., Pearce, O., and Morris, K.
(2018). Objective assessment of stiffness in Achilles tendinopathy: a
novel approach using the MyotonPRO. \emph{BMJ Open Sport \& Exercise
Medicine} 4, e000446.
doi:\href{https://doi.org/10.1136/bmjsem-2018-000446}{10.1136/bmjsem-2018-000446}.

\leavevmode\hypertarget{ref-Murphy2003}{}%
Murphy, D. F., Connolly, D. A. J., and Beynnon, B. D. (2003). Risk
factors for lower extremity injury: A review of the literature.
\emph{British Journal of Sports Medicine} 37, 13--29.
doi:\href{https://doi.org/10.1136/bjsm.37.1.13}{10.1136/bjsm.37.1.13}.

\leavevmode\hypertarget{ref-Murphy2002}{}%
Murphy, K. (2002). Dynamic Bayesian networks: Representation, inference
and learning.

\leavevmode\hypertarget{ref-Nair2014}{}%
Nair, K., Dougherty, J., Schaefer, E., Kelly, J., and Masi, A. (2014).
Repeatability, reproducibility, and calibration of the MyotonPRO on
tissue mimicking phantoms. in \emph{ASME summer bioengineering
conference}, 1--2.
doi:\href{https://doi.org/10.1115/SBC2013-14622}{10.1115/SBC2013-14622}.

\leavevmode\hypertarget{ref-Neely1998}{}%
Neely, F. G. (1998). Biomechanical risk factors for exercise-related
lower limb injuries. \emph{Sports Medicine} 26, 395--413.
doi:\href{https://doi.org/10.2165/00007256-199826060-00003}{10.2165/00007256-199826060-00003}.

\leavevmode\hypertarget{ref-Olmedilla2018}{}%
Olmedilla, A., Rubio, V. J., Fuster-Parra, P., Pujals, C., and
García-Mas, A. (2018). A Bayesian approach to sport injuries likelihood:
Does player's self-efficacy and environmental factors plays the main
role? \emph{Frontiers in Psychology} 9, 1--10.
doi:\href{https://doi.org/10.3389/fpsyg.2018.01174}{10.3389/fpsyg.2018.01174}.

\leavevmode\hypertarget{ref-Passer1983a}{}%
Passer, M. W., and Seese, M. D. (1983). Life stress and athletic injury:
Examination of positive versus negative events and three moderator
variables. \emph{Journal of Human Stress} 9, 11--16.
doi:\href{https://doi.org/10.1080/0097840X.1983.9935025}{10.1080/0097840X.1983.9935025}.

\leavevmode\hypertarget{ref-Pearl1988}{}%
Pearl, J. (1988). \emph{Probabilistic reasoning in intelligent systems:
Networks of plausible inference}. San Francisco, CA: Morgan Kaufmann
Publishers.

\leavevmode\hypertarget{ref-Perna2003}{}%
Perna, F. M., Antoni, M. H., Baum, A., Gordon, P., and Schneiderman, N.
(2003). Cognitive behavioral stress management effects on injury and
illness among competitive athletes: A randomized clinical trial.
\emph{Annals of Behavioral Medicine} 25, 66--73. Available at:
\url{https://www.scopus.com/inward/record.uri?eid=2-s2.0-0037262738\%7B/\&\%7DpartnerID=40\%7B/\&\%7Dmd5=37332ebe6a962abf5178e51d6ebaedb7}.

\leavevmode\hypertarget{ref-Perna1995}{}%
Perna, F. M., and McDowell, S. L. (1995). Role of psychological stress
in cortisol recovery from exhaustive exercise among elite athletes.
\emph{International Journal of Behavioral Medicine} 2, 13--26.
doi:\href{https://doi.org/10.1207/s15327558ijbm0201_2}{10.1207/s15327558ijbm0201\_2}.

\leavevmode\hypertarget{ref-Perna1997}{}%
Perna, F., Schneiderman, N., and LaPerriere, A. (1997). Psychological
stress, exercise and immunity. \emph{International Journal of Sports
Medicine} 18, 78--83.
doi:\href{https://doi.org/10.1055/s-2007-972703}{10.1055/s-2007-972703}.

\leavevmode\hypertarget{ref-Petrie1992}{}%
Petrie, T. A. (1992). Psychosocial antecedents of athletic injury: The
effects of life stress and social support on female collegiate gymnasts.
\emph{Journal of Behavioral Medicine} 18, 127--138.
doi:\href{https://doi.org/10.1080/08964289.1992.9936963}{10.1080/08964289.1992.9936963}.

\leavevmode\hypertarget{ref-Plews2017}{}%
Plews, D. J., Scott, B., Altini, M., Wood, M., Kilding, A. E., and
Laursen, P. B. (2017). Comparison of heart-rate-variability recording
with smartphone photoplethysmography, polar H7 chest strap, and
electrocardiography. \emph{International Journal of Sports Physiology
and Performance} 12, 1324--1328.
doi:\href{https://doi.org/10.1123/ijspp.2016-0668}{10.1123/ijspp.2016-0668}.

\leavevmode\hypertarget{ref-Pruyn2015}{}%
Pruyn, E. C., Watsford, M. L., and Murphy, A. J. (2015). Differences in
lower-body stiffness between levels of netball competition.
\emph{Journal of Strength and Conditioning Research} 29, 1197--1202.
doi:\href{https://doi.org/10.1519/JSC.0000000000000418}{10.1519/JSC.0000000000000418}.

\leavevmode\hypertarget{ref-Pruyn2016}{}%
Pruyn, E. C., Watsford, M. L., and Murphy, A. J. (2016). Validity and
reliability of three methods of stiffness assessment. \emph{Journal of
Sport and Health Science} 5, 476--483.
doi:\href{https://doi.org/10.1016/j.jshs.2015.12.001}{10.1016/j.jshs.2015.12.001}.

\leavevmode\hypertarget{ref-RCoreTeam2019}{}%
R Core Team (2019). \emph{R: A language and environment for statistical
computing}. Vienna, Austria: R Foundation for Statistical Computing
Available at: \url{https://www.r-project.org/}.

\leavevmode\hypertarget{ref-Riemann1999d}{}%
Riemann, B. L., Guskiewicz, K. M., and Shields, E. W. (1999).
Relationship between clinical and forceplate measures of postural
stability. \emph{Journal of Sport Rehabilitation} 8, 71--82.
doi:\href{https://doi.org/10.1123/jsr.8.2.71}{10.1123/jsr.8.2.71}.

\leavevmode\hypertarget{ref-Rogers2005}{}%
Rogers, T., and Landers, D. M. (2005). Mediating effects of peripheral
vision in the life event stress/athletic injury relationship.
\emph{Journal of Sport and Exercise Psychology} 27, 271--288.
doi:\href{https://doi.org/10.1002/9781444303650}{10.1002/9781444303650}.

\leavevmode\hypertarget{ref-Romero-Franco2014}{}%
Romero-Franco, N., Gallego-Izquierdo, T., Martínez-López, E. J.,
Hita-Contreras, F., Osuna-Pére, Catalina, M., and Martínez-Amat, A.
(2014). Postural stability and subsequent sports injuries during indoor
season of athletes. \emph{Journal of Physical Therapy Science} 26,
683--687.
doi:\href{https://doi.org/10.1589/jpts.26.683}{10.1589/jpts.26.683}.

\leavevmode\hypertarget{ref-Rosa2014}{}%
Rosa, B. B., Asperti, A. M., Helito, C. P., Demange, M. K., Fernandes,
T. L., and Hernandez, A. J. (2014). Epidemiology of sports injuries on
collegiate athletes at a single center. \emph{Acta Ortopédica
Brasileira} 22, 321--324.
doi:\href{https://doi.org/10.1590/1413-78522014220601007}{10.1590/1413-78522014220601007}.

\leavevmode\hypertarget{ref-Norvig2009}{}%
Russell, S. J., and Norvig, P. (2009). \emph{Artificial Intelligence: A
Modern Approach}. 3rd ed. Prentice Hall Available at:
\url{http://aima.cs.berkeley.edu/}.

\leavevmode\hypertarget{ref-Scutari2017}{}%
Scutari, M., Auconi, P., Caldarelli, G., and Franchi, L. (2017).
Bayesian networks analysis of malocclusion data. \emph{Scientific
Reports} 7, 1--11.
doi:\href{https://doi.org/10.1038/s41598-017-15293-w}{10.1038/s41598-017-15293-w}.

\leavevmode\hypertarget{ref-Scutari2014}{}%
Scutari, M., and Denis, J.-B. (2014). \emph{Bayesian networks: with
examples in R}. 1st ed. Chapman \& Hall/CRC Available at:
\url{https://www.crcpress.com/Bayesian-Networks-With-Examples-in-R/Scutari-Denis/p/book/9781482225587}.

\leavevmode\hypertarget{ref-Scutari2013}{}%
Scutari, M., and Nagarajan, R. (2013). Identifying significant edges in
graphical models of molecular networks. \emph{Artificial Intelligence in
Medicine} 57, 207--217.
doi:\href{https://doi.org/10.1016/j.artmed.2012.12.006}{10.1016/j.artmed.2012.12.006}.

\leavevmode\hypertarget{ref-Sheu2016}{}%
Sheu, Y., Chen, L. H., and Hedegaard, H. (2016). Sports- and
recreation-related injury episodes in the United States, 2011-2014.
\emph{National Health Statistics Reports}, 1--12. Available at:
\url{https://www.ncbi.nlm.nih.gov/pubmed/27906643}.

\leavevmode\hypertarget{ref-Slimani2018}{}%
Slimani, M., Bragazzi, N. L., Znazen, H., Paravlic, A., Azaiez, F., and
Tod, D. (2018). Psychosocial predictors and psychological prevention of
soccer injuries: A systematic review and meta-analysis of the
literature. \emph{Physical Therapy in Sport} 32, 293--300.
doi:\href{https://doi.org/10.1016/j.ptsp.2018.05.006}{10.1016/j.ptsp.2018.05.006}.

\leavevmode\hypertarget{ref-Smith1990}{}%
Smith, R. E., Smoll, F. L., and Ptacek, J. T. (1990). Conjunctive
moderator variables in vulnerability and resiliency research: Life
stress, social support and coping skills, and adolescent sport injuries.
\emph{Journal of Personality and Social Psychology} 58, 360--370.
doi:\href{https://doi.org/10.1037/0022-3514.58.2.360}{10.1037/0022-3514.58.2.360}.

\leavevmode\hypertarget{ref-Swanik2007}{}%
Swanik, C. B., Covassin, T., Stearne, D. J., and Schatz, P. (2007). The
relationship between neurocognitive function and noncontact anterior
cruciate ligament injuries. \emph{American Journal of Sports Medicine}
35, 943--948.
doi:\href{https://doi.org/10.1177/0363546507299532}{10.1177/0363546507299532}.

\leavevmode\hypertarget{ref-Timpka2014}{}%
Timpka, T., Alonso, J. M., Jacobsson, J., Junge, A., Branco, P.,
Clarsen, B., Kowalski, J., Mountjoy, M., Nilsson, S., Pluim, B., et al.
(2014). Injury and illness definitions and data collection procedures
for use in epidemiological studies in Athletics (track and field):
Consensus statement. \emph{British Journal of Sports Medicine} 48,
483--490.
doi:\href{https://doi.org/10.1136/bjsports-2013-093241}{10.1136/bjsports-2013-093241}.

\leavevmode\hypertarget{ref-Valovich2003}{}%
Valovich, T. C., Perrin, D. H., and Gansneder, B. M. (2003). Repeat
administration elicits a practice effect with the Balance Error Scoring
System but not with the Standardized Assessment of Concussion in high
school athletes. \emph{Journal of Athletic Training} 38, 51--56.
Available at: \url{https://www.ncbi.nlm.nih.gov/pubmed/12937472}.

\leavevmode\hypertarget{ref-Wiese-Bjornstal2009}{}%
Wiese-Bjornstal, D. M. (2009). Sport injury and college athlete health
across the lifespan. \emph{Journal of Intercollegiate Sport} 2, 64--80.
doi:\href{https://doi.org/10.1123/jis.2.1.64}{10.1123/jis.2.1.64}.

\leavevmode\hypertarget{ref-Wilkerson2012a}{}%
Wilkerson, G. B. (2012). Neurocognitive reaction time predicts lower
extremity sprains and strains. \emph{International Journal of Athletic
Therapy and Training} 17, 4--9.
doi:\href{https://doi.org/10.1123/ijatt.17.6.4}{10.1123/ijatt.17.6.4}.

\leavevmode\hypertarget{ref-Williams2007}{}%
Williams, J. M., and Andersen, M. B. (2007). ``Psychosocial antecedents
of sport injury and interventions for risk reduction,'' in
\emph{Handbook of sport psychology}, eds. G. Tenenbaum and R. C. Eklund
(Hoboken, NJ, USA: Wiley), 379--403.
doi:\href{https://doi.org/10.1002/9781118270011.ch17}{10.1002/9781118270011.ch17}.

\leavevmode\hypertarget{ref-Williams1998}{}%
Williams, J. M., and Andersen, M. B. (1998). Psychosocial antecedents of
sport injury: review and critique of the stress and injury model.
\emph{Journal of Applied Sport Psychology} 10, 5--25.
doi:\href{https://doi.org/10.1080/10413209808406375}{10.1080/10413209808406375}.

\leavevmode\hypertarget{ref-Williams2017}{}%
Williams, S., Booton, T., Watson, M., Rowland, D., and Altini, M.
(2017). Heart rate variability is a moderating factor in the
workload-injury relationship of competitive crossfit™ athletes.
\emph{Journal of Sports Science and Medicine} 16, 443--449.

\leavevmode\hypertarget{ref-Xiao-xuan2007}{}%
Xiao-xuan, H., Hui, W., and Shuo, W. (2008). Using expert's knowledge to
build Bayesian networks. in \emph{International conference on
computational intelligence and security workshops} (IEEE), 220--223.
doi:\href{https://doi.org/10.1109/cisw.2007.4425484}{10.1109/cisw.2007.4425484}.

\leavevmode\hypertarget{ref-Xu2015}{}%
Xu, J. G., Zhao, Y., Chen, J., and Han, C. (2015). A structure learning
algorithm for Bayesian network using prior knowledge. \emph{Journal of
Computer Science and Technology} 30, 713--724.
doi:\href{https://doi.org/10.1007/s11390-015-1556-8}{10.1007/s11390-015-1556-8}.

\end{document}
